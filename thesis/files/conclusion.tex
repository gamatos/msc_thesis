\documentclass[notas.tex]{subfiles}

\begin{document}
\chapter{Conclusions and further work}\label{sec_conclusion}
For single particle states and integer quantum Hall states, complex-time evolution does not significantly alter the wave function structure in the examples considered here. Indeed, after normalization, $e^{-s\q{H}}$ acts as the identity, as one can see in \eqref{eq_plane_spf_evolution} and \eqref{eq_plane_iqh_evolution} for the plane, and in \eqref{eq_cyl_spf_evolution} and \eqref{eq_cyl_iqh_evolution} for the cylinder. In accordance to \remref{rem_gcst_interpretation}, this means that the change of the wave function as the surface is deformed corresponds to a direct lift of the classical evolution. Indeed, taking into account the deformation of the geometry of the system, the evolved wave functions have the same structure as the initial wave function. 

In the case of fractional quantum Hall states, however, while the action of $e^{s\prq{H}}$ is similar to that in the case of integer quantum Hall states (see \eqref{mr_evolved}), the action of $e^{-s\q{H}}$ is no longer trivial due to the powers $m$ appearing in the holomorphic factor. 

Indeed, this action leads to changes in the structure of the wave function, as seen in \eqref{eq_prefactor_change_plane} for the plane and in \eqref{eq_prefactor_change_cyl} for the cylinder. This difference might be reflecting the Coulomb interaction among electrons, which one must take into account in fractional quantum Hall systems and which is being affected by the deformation of the surface. This interpretation is suggested by the fact that the power in the holomorphic factor of the wave function is related to the fractionalization of the charge of the electron, which is a phenomenon arising precisely due to the Coulomb interactions. Further analysis of this change in wave function stucture is needed. 

The change in the structure of the wave function is even more pronounced in the case of the Moore-Read wave function, since the Pfaffian factor makes it so that there are different powers appearing in
the holomorphic factor for different terms (see \eqref{eq_pfaffian_structure_change}).

Note that, while \eqref{eq_laughlin_wavefunction} and \eqref{eq_laughlin_cyl} have zeros of order $m$ when the position of exactly two particles coincide, the change of structure just discussed alters this substantially. This also happens in \cite{qiu_model_2012} (see Fig. 1 and the discussion following it), which contradicts (2) of Definition 2 of \cite{klevtsov_laughlin_2019}, which demands that this property is kept for all geometries. 

As for quasiholes, there is effectively a change in the position of the quasihole depending on the deformation parameter $s$, as seen in \eqref{eq_qh_change}. 


There is an interesting phenomenon arising when one takes the limit $s \to \infty$. For instance, in the case of the cylinder, as \figref{fig_spf_cyl_evolution} and \figref{fig_spf_cyl_evolution_sec} suggest, and is explicitely seen in \eqref{eq_cyl_iqh_evolution}, the wave functions approximate distributional states concentrated around the Bohr-Sommerfeld leaves $u = 2 \pi m$. 

Mathematically, the justification for the use of GCSTs comes from the fact that these transforms are unitary in some particular cases (see \cite{kirwin_complex_2013}) and asymptotically unitary for more general cases (see \thmref{thm_gcst_asymp_unitarity}). The use of complex Hamiltonian flows to deform the surface comes from the fact that, for $\tau = is, s \in \reals^+$, they yield geodesics in the space of Kähler metrics (see \thmref{thm_complex_flow_geodesics}).

Overall, the approach presented here has the advantage of being able to make theoretical predictions for arbitrarily big deformations of the surface, corresponding to the parameter $s$. This contrasts with the perturbative techniques used e.g. in \cite{johri_probing_2016}, where the deformation must be very small. 

Further work is needed to apply these techniques to different geometies eg. the sphere (whose Laughlin wave function was given by Haldane in \cite{haldane_fractional_1983}) and the torus (whose Laughlin wave function was given by Haldane in \cite{haldane_periodic_1985}). An attempt to do so for the case of the torus was made; however, fully quantizing the Hamiltonian as in \remref{rem_full_quantization} turned out to be difficult. A possible solution may be to use Berezin-Toeplitz quantization (see e.g. \cite{englis_excursion_2016}).

It is also important to perform experimental and numerical tests of the adequateness of the proposed dependence of quantum Hall ground states on the geometry of the surface.
\end{document}