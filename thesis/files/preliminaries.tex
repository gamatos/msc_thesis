\documentclass[notas.tex]{subfiles}

\begin{document}
\chapter{Preliminaries} \label{sec_prelim}

In these preliminaries, we present the theory needed in the following sections. We will assume that the reader is familiar with basic differential and Riemannian geometry. Very good references for these topics are \cite{godinho_introduction_2014, boothby_introduction_1986,do_carmo_riemannian_1992} and \cite{kobayashi_foundations_1963, kobayashi_differential_1987}; the latter are particularly good as references for fiber bundles. For physicists, \cite{nakahara_geometry_2003} is a popular reference for these subjects; other references for those with a physical background are \cite{isham_modern_1999, nash_topology_1983}. 

The reader is also assumed to be familiar with elementary quantum mechanics; \cite{griffiths_introduction_2018} is a standard reference in physics. For a mathematical point of view, \cite{hall_quantum_2013} is an excellent reference.

\section{Geometric Mechanics} \label{sec_gm}

In this section, we will lay the geometric foundations that we will need in what follows.

We begin by establishing some notation. We will often denote smooth manifolds by $\mfld$ and refer to them simply as manifolds. The dimension of a manifold $\mfld$ will be denoted by $\dim \mfld$. We will denote the smooth atlas associated to $\mfld$ as $\atlas$, which we assume is made up of pairs $(U, \phi)$, $\phi: U \to \reals^{\dim \mfld}$ being a chart defined on $U \subseteq \mfld$. By abuse of notation, we write $(U, (x_1,..., x_n)) \in \atlas$ simply as $(U, x_1,..., x_n) \in \atlas$.

Given a manifold $\mfld$, we will denote its tangent bundle by $T\mfld$ and its cotangent bundle by $T^*\mfld$.

We denote the set of smooth functions from the manifold $\mfld$ to the manifold $\mathcal{N}$ by $C^\infty(\mfld; \mathcal{N})$, where $\mathcal{N}$ will often be either $\cmplx$ or $\reals$ with the usual differentiable structure; in the latter case, we simply write $C^\infty(\mfld)$. We denote the set of smooth vector fields on $\mfld$ by $\vf(\mfld)$ and
the set of smooth differential $n$-forms on $\mfld$ by $\dform^n(\mfld)$. Their real analytic counterparts will be denoted by $C^\ra(\mfld; \mathcal{N}), \vf^\ra(\mfld)$ and $\dform^{n,\ra}(\mfld)$, respectively. Given $X \in \vf(\mfld)$ and $\nu \in \dform^{k}(\mfld)$, we denote the inner derivative of $\nu$ by $X$ as $\contr{X} \nu$. Given $X \in \vf(\mfld)$, we denote the flow of $X$ at time $t \in \reals$ by $\phi^X_t$, or simply by $\phi_t$ when $X$ is clear from context.

% As usual, given a Lie group $G$ with identity $e$ and a manifold $\mfld$, a (left) smooth action is a smooth function $\act: G \times \mfld \to \mfld$ satisfying, for all $g, h \in G$ and $p \in \mfld$,
% \begin{align*}
% 	\act(gh)(p) &= \act(g)(\act(h)(p)), \\
% 	\act(e)(p) &= p. \\
% \end{align*}

% Let $\act$ be a smooth action of a Lie group $G$ on a manifold $\mfld$. Given a vector field in the Lie algebra $\lalg$ of $\lgr$, we denote by $X^\act$ the vector field on $\mfld$ generated by the usual Lie group exponential map $\exp(t X)$. We denote by $\Ad$ and $\Ad^*$ the usual adjoint and coadjoint actions, respectively. \\

Given a manifold $\mfld$, we denote by $H^k(\mfld; \reals)$ its $k$th de Rham cohomology group, and given a closed differential $k$-form $\nu \in \dform^k(\mfld)$, we denote its de Rham cohomology class by $\left[ \nu \right] \in H^k(\mfld; \reals)$.

% We will need the well known Poincaré lemma
% \begin{lem}[Poincaré] \label{lem_poincare}
% 	\begin{align*}
% 		H^k(\reals^d; \reals) = \begin{cases}
% 			\reals, &\text{if } k=0 \\
% 			0, &\text{if } k \neq 0
% 		\end{cases}
% 	\end{align*}
% \end{lem}



% is such that for $X_1,...,X_{k-1} \in \sect(T \mfld)$,
% \begin{align*}
% 	(\contr{X} \nu)(X_1,...,X_{k-1}) := \nu(X, X_1, ..., X_{k-1})
% \end{align*}

We will also need the concept of the musical isomorphism.
\begin{prop}\label{prop_musical_iso}
	Let $\mfld$ be a manifold and $\nu \in \dform^2(\mfld)$ be a nondegenerate differential $2$-form. Then, for $\alpha \in \dform^1(\mfld)$, there is a unique vector field $X_\alpha \in \vf(\mfld)$ such that 
	\begin{align*}
		\contr{X_\alpha} \nu = \alpha.
	\end{align*}  
	The map $\cdot^{\#_\nu}: \dform(\mfld) \to \vf(\mfld)$ such that $\alpha^{\#_\nu} = X_\alpha$ is linear and bijective, and is said to be the \emph{musical isomorphism associated to $\nu$}.
\end{prop}

Given a real vector bundle $\vbund$ of rank $k$ on a manifold $\mfld$, we will denote by $\batlas$ the associated collection of vector bundle trivializing charts, which we assume is made up of pairs $(U, \phi)$, with $\phi: U \times \reals^{\dim \mfld} \to E$ and $U \subseteq \reals^k$. 

Given a vector bundle $\vbund$ over a manifold $\mfld$, we denote its set of smooth sections by $\sect(E)$ and its complexification by $\vbund^\cmplx$. Given $U \subseteq \mfld$ an open set, we denote by $\sect(E)|_U$ the set of smooth sections defined on $U$. We will, in particular, define $\vf(\mfld; \cmplx) := \sect(T\mfld^\cmplx)$ and $\dform^n(\mfld; \cmplx) := \sect(\wedge^n T^*\mfld^\cmplx)$, where $\wedge$ is the usual exterior product of vector bundles.

We say that $R$ is a vector subbundle of $\vbund$, and denote this by $R \subseteq \vbund$, if $R$ is a vector bundle over $\mfld$ such that $R(p) \subseteq \vbund(p)$ for all $p \in \mfld$. Moreover, if $R,S \subseteq E$ are two vector subbundles of a vector bundle $E$, we denote by $R \cap S$ the fiberwise intersection of $R$ and $S$. We denote the complex conjugate of a complex vector bundle $\vbund$ by $\overline E$. \\

\subsection{Sympectic geometry} \label{sec_sg}

Here, we will introduce the symplectic geometry formalism used to describe Hamiltonian dynamics. For references on symplectic geometry, see \cite{cannas_da_silva_lectures_2001, mcduff_introduction_2017}.

\begin{defn}
	A \emph{symplectic manifold} is a pair $(\mfld,\sform)$, where $\mfld$ is a differentiable manifold and $\sform \in \dform^{2}(\mfld)$ is a closed, non-degenerate differential $2$-form called the \emph{symplectic form}.
\end{defn}
\begin{ex}\label{ex_cotangent_bundle}
	The cotangent bundle $\mfld = T^*\nfld$ of a manifold $\nfld$ with the canonical symplectic form $\sform = d \spot$, where $\spot$ is the tautological form (see \cite[Ch. 2]{cannas_da_silva_lectures_2001}).
\end{ex}

\begin{lem}\label{lem_sform_even_dim}
	Let $\mfld$ be a manifold. If $\sform$ is a symplectic form defined on $\mfld$, then $\dim \mfld = 2n$ for some $n \in \nats$.
\end{lem}

\begin{defn}
	Let $(\mfld, \sform), (\mfldb, \sform')$ be two symplectic manifolds. A \emph{symplectomorphism} is a diffeomorphism $\phi: \mfld \to \mfldb$ such that $\sform = \phi^* \sform'$.
\end{defn}

\begin{prop}
	Let $(\mfld, \sform)$ be a $2n$-dimensional symplectic manifold. The $2n$-form
	\begin{align*}
		\frac{\sform^n}{n!} \in \dform^{2n}(\mfld)
	\end{align*}
	defines a volume form on $\mfld$ and is called the \emph{Liouville form} or the \emph{symplectic volume} of $\mfld$.
\end{prop}

The Darboux theorem, which we now state, says that locally all symplectic manifolds can be identified with an open subset of $T^* \reals^n \cong \reals^{2n}$ with the canonical symplectic form (and thus are locally indistinguishable as symplectic manifolds). 

\begin{thm}[Darboux]\label{thm_darboux_coord}
	Let $(\mfld, \sform)$ be a symplectic manifold. Then, for any $p \in \mfld$ there exists a chart $(U, x_1,...,x_n,y_1,...,y_n) \in \atlas$ centered at $p$ such that, on $U$,
	\begin{align*}
		\sform = \sum_{j=1}^n d x_j \wedge d y_j.
	\end{align*}
\end{thm}
% We denote by $\vf^\ra(\mfld)$ the Lie algebra of all real analytic vector fields defined on $\mfld$, and by $C^\ra(\mfld)$ the Lie algebra (with respect to the Poisson bracket) of all real analytic functions defined on $\mfld$. 
\begin{defn}
	Given a function $f \in C^\infty(\mfld)$, its \emph{Hamiltonian vector field} is the vector field $X_f = (-df)^{\#_\sform} \in C^\infty(T \mfld)$, i.e. the unique vector field which satisfies
	\begin{align*}
		\contr{X_f}{\sform} = - df.
	\end{align*}
\end{defn}

\begin{defn}
	Let $(\mfld, \sform)$ be a symplectic manifold, $\lgr$ be a Lie group and $\lalg$ be its Lie algebra. A smooth action $\hact: G \to \diff(\mfld)$ is said to be \emph{Hamiltonian} if there exists a map
	\begin{align*}
		\mmap: \mfld \to \lalg^*
	\end{align*} 
	such that
	\begin{enumerate}
		\item For $X \in \lalg$, it is the case that
		\begin{align*}
			d \mmap(X) = - \contr{X^{\hact}} \sform,
		\end{align*}
		where $X^\hact$ is the fundamental vector field corresponding to $X$.
		%  and 
		% \begin{align*}
		% 	\mmap^*: \lalg \to C^\infty(\reals)
		% \end{align*} 
		% is the \emph{comoment map}, defined for $p \in \mfld$ as $\mmap^*(X)(p) := \mmap(p)(X)$.
		\item $\mmap$ satisfies
		\begin{align*}
			\mmap \circ \hact(g) = \Ad^*(g) \circ \mmap,
		\end{align*}
		where $\Ad^*$ denotes the usual coadjoint action of $\lgr$ on $\lalg^*$.
	\end{enumerate}
\end{defn}

\subsection{Complex geometry} \label{sec_complex}
Here, we briefly present some concepts related to complex geometry, which can be thought as a generalization of complex analysis on $\cmplx^n$ and its results to manifolds. 
\begin{defn}\label{def_complex_mfld}
	A complex manifold $\mfld$ is a $2n$-dimensional manifold with a choice of subset $\atlas^\holo \subseteq \atlas$ such that, for all $(U,\phi), (V,\psi) \in \atlas^\holo$, $\xi \circ \psi \circ \phi^{-1} \circ \xi^{-1}$ is holomorphic, where
	\begin{align*}
		\xi: \reals^{2n} &\to \cmplx^n \\
		(a_1,...,a_n,b_1,...,b_n) &\mapsto (a_1 - i b_1,...,a_n-ib_n),
	\end{align*}
	and such that $\cup \, \{U \subseteq \mfld : (U,\phi) \in \atlas^\holo\} = \mfld$. We call such a subset a \emph{holomorphic atlas} and a \emph{complex structure} if it is maximal in the family of holomorphic atlases. We call the elements of $\atlas^\holo$ \emph{holomorphic charts}. For $(U,x_1,...,x_n,y_1,...,y_n) \in \atlas^\holo$, we define $(z_1,...,z_n) := \xi(x_1,...,x_n,y_1,...,y_n)$ and call $(U, z_1,...,z_n)$ a \emph{local set of holomorphic coordinates}.
\end{defn}

Under the above identification, we define
\begin{align} \label{eq_holo_vecs}
	\pd{z_j} &:= \frac{1}{2} \left( \pd{x_j} + i \pd{y_j} \right), \\
	\pd{\bar{z}_j} &:= \frac{1}{2} \left( \pd{x_j} - i \pd{y_j} \right). \nonumber
\end{align}

Note that, analogously to the differentiable case, every holomorphic atlas is contained in a unique maximal holomorphic atlas.

\begin{defn} \label{def_holo}
	Let $\mfld$ be a $2n$-dimensional complex manifold with holomorphic atlas $\atlas^\holo$ and let $V \subseteq \mfld$ be an open set.

	A function $f \in C^\infty(V; \cmplx)$ is said to be \emph{holomorphic} (on $V$) if, for any holomorphic chart $(U, \phi) \in \atlas^\holo$, $f \circ \phi^{-1} \circ \xi^{-1}$ is holomorphic on $(\xi \circ \phi) (U \cap V)$, where $\xi$ is as in \defref{def_complex_mfld}. We denote the set of such functions by $C^\holo(V)$.

	A vector field $X \in \vf(V; \cmplx)$ is said to be \emph{holomorphic} (on $V$) if, for any local set of holomorphic coordinates $(U, z_1, ..., z_n)$, $X|_{(U \cap V)} = \sum_{i=1}^{n} f_i \pd{z_i}$ for some $f_i \in C^\holo(U \cap V), i \in \{1,...,n\}$. We denote the set of holomorphic vector fields (on $V$) by $\vf^\holo(V)$. 
	
	 A differential $k$-form $\eta \in \dform^k(\mfld; \cmplx)$ is holomorphic (on $V$) if, for any local set of holomorphic coordinates $(U, z_1, ..., z_n)$, $\eta|_{(U \cap V)} = \sum_{1 \leq i_1 < ... < i_k \leq n} f_{i_1...i_k} dz_{i_1} \wedge ... \wedge  dz_{i_k}$ for some $f_{i_1,...,i_k} \in C^\holo(U \cap V)$. We denote the set of holomorphic $k$-forms (on $V$) by $\dform^{k,\holo}(V)$.
\end{defn}

\begin{defn} \label{def_cstruct}
	Given a manifold $\mfld$, an almost complex structure $\cstruct$ is a bundle isomorphism $\cstruct: T \mfld \to T \mfld$ such that $\cstruct^2 = -I$, where $I$ is the identity map on $T \mfld$.
\end{defn}

\begin{lem}\label{lem_cstruct_even_dim}
	Let $\mfld$ be a manifold. If $\mfld$ admits an almost complex structure $\cstruct$, then $\dim \mfld = 2n$ for some $n \in \nats$.
\end{lem}

\begin{prop} \label{prop_cstruct_integrable}
	Let $\mfld$ be a manifold and $\cstruct$ be an almost complex structure on $\mfld$. By \lemref{lem_cstruct_even_dim}, $\dim \mfld = 2n$ for some $n \in \nats$. Suppose that $\forall p \in \mfld$, there exists a chart $(U, x_1,...,x_n,y_1,...,y_n) \in \atlas$ with $p \in U$ such that
	\begin{align} \label{eq_complex_str_def}
		J \pd{x_i} &= \pd{y_i}, \\
		J \pd{y_i} &= - \pd{x_i}. \nonumber
	\end{align}
	Then, the set of such charts defines a complex structure on $\mfld$ in the sense of \defref{def_complex_mfld} and we say that $\cstruct$ is \emph{integrable}.
	Conversely, let $\mfld$ be a complex manifold. Then there is an unique almost complex structure $J: T\mfld \to T\mfld$ such that for all $(U, x_1,...,x_n,y_1,...,y_n) \in \atlas^\holo$, $J$ satisfies \eqref{eq_complex_str_def}.
\end{prop}
\begin{prop}
	Let $\mfld$ be a manifold and $J$ be an almost complex structure defined on $\mfld$. The map $N_J : \vf(\mfld) \times \vf(\mfld) \to \vf(\mfld)$, defined for $X,Y \in \vf(\mfld)$ as
	\begin{align*}
		N_J(X,Y) = [JX,JY] - J[X,JY] - J[JX,Y] - [X,Y]
	\end{align*}
	is a tensor field and is called the \emph{Nijenhuis tensor} associated to $J$.
\end{prop}
\begin{thm}[Newlander-Nirenberg \cite{newlander_complex_1957}]
	Let $\mfld$ be a manifold and $J$ be an almost complex structure defined on $T \mfld$. Then  $J$ is integrable if and only if $N_J \equiv 0$.
\end{thm}

In light of \propref{prop_cstruct_integrable}, by an abuse of terminology, we sometimes call an integrable almost complex structure $J$ simply a complex structure and treat it interchangeably with the actual complex structure defined by $\atlas^\holo$.

% In the previous definition, we could have chosen the opposite signs e.g. we could have defined $z_j = x_j + i y_j$ instead. The choice above is, however, necessary since it has important implications on geometric quantization (see \secref{sec_gq}). It also has the effect of determining the direction of the magnetic field (see \secref{sec_gq_particle_magnetic}).

\begin{defn}\label{def_cstruct_constructions}
	Let $\mfld$ be a manifold and $J$ an almost complex structure defined on it. The set of all $v \in T\mfld^\cmplx$ safisfying $Jv = iv$ is denoted by $T_{1,0}$, and the set of all $v \in T\mfld^\cmplx$ safisfying $Jv = -iv$ is denoted by $T_{0,1}$.	
	
	Similarly, the set of all $\alpha \in T^*\mfld^\cmplx$ such that $J^* \alpha = i \alpha$ is denoted by $T^{1,0}$, and the set of all $\alpha \in T^*\mfld^\cmplx$ satisfying $J^* \alpha = -i \alpha$ is denoted by $T^{0,1}$. We write $\Lambda^k(T^*\mfld^\cmplx) = \oplus_{l+m=k} \Lambda^{l,m}$, where $\Lambda^{l,m}:= (\Lambda^l T^{1,0}) \wedge (\Lambda^m T^{0,1})$. We define $\dform^{l,m}(\mfld) := \sect(\Lambda^{l,m})$ and note that $\dform^k(\mfld; \cmplx) = \oplus_{l+m = k} \Omega^{l,m}(\mfld)$. We refer to the elements of $\dform^{l,m}(\mfld)$ as \emph{$(l,m)$-forms on $\mfld$}.
	
	We define the Dolbeaut operators
	\begin{align*}
		\hd&:= \pi^{l+1,m} \circ d: \dform^{l, m}(\mfld) \to \dform^{l+1, m}(\mfld), \\
		\ahd&:= \pi^{l,m+1} \circ d: \dform^{l, m}(\mfld) \to \dform^{l, m+1}(\mfld),
	\end{align*}
	where $\pi^{l,m}$ is the projection onto $\Lambda^{l,m}$.
\end{defn}

\begin{prop}\label{prop_nn_ispace}
	Let $\mfld$ be a manifold and $J$ be an almost complex structure defined on $T\mfld$. Then $N_J \equiv 0$ if and only if, for all $X,Y \in T_{1,0}$, $\lb{X}{Y} \in T_{1,0}$, where $T_{1,0}$ is as in \defref{def_cstruct_constructions}.
\end{prop}

Given two complex manifolds $\mfld, \mfldb$, a \emph{biholomorphism} is a holomorphic diffeomorphism; or, equivalently, a diffeomorphism $\phi: \mfld \to \mfldb$ such that $\phi^*J_\mfldb = J_\mfld$, where $J_\mfld, J_\mfldb$ are the complex structures associated to $\mfld$ and $\mfldb$, respectively. 

\defref{def_holo} has an equivalent, coordinate-independent formulation in terms of the structures in \defref{def_cstruct_constructions}.
\begin{prop}
	Let $\mfld$ be a complex manifold with complex structure $J$ and $U \subseteq \mfld$ an open set. Then
	\begin{enumerate}
		\item $X \in \vf^\holo(U)$ if and only if $X \in \sect(T_{1,0})|_U$ and $df(X) \in C^\holo(U)$ for all $f \in C^\holo(V)$ with $V \subseteq U$ open.
		\item $\eta \in \dform^{k, \holo}(U)$ if and only if $\eta \in \sect(\Lambda^{k,0})|_U$ and $\ahd \eta = 0$.
		\item $f \in C^\holo(U)$ if and only if $df$ is holomorphic.
	\end{enumerate}
\end{prop}

Note that, for a complex manifold $\mfld$ with complex structure $J$, by \defref{def_complex_mfld} and \propref{prop_cstruct_integrable}, $J: T\mfld \to T\mfld$ can be written in a local set of holomorphic coordinates $(U, z_1,...,z_n)$ as
\begin{equation} \label{structure_from_coordinates}
	\cstruct = \sum_{j=1}^n i \frac{\partial}{\partial z_j} \otimes dz_j - i \frac{\partial}{\partial \bar z_j} \otimes d \bar z_j.
\end{equation}
% When we want make the integrable almost complex structure (and the associated complex structure) explicit, we may say that e.g. $f \in C^\holo(\mfld)$ is $J$-holomorphic instead of just holomorphic. \\

\subsection{Kähler geometry} \label{sec_kahler}

Kähler geometry lies at the intersection of symplectic, complex, and Riemannian geometries. As such, it provides us with tools coming from all three theories and their combination. Introductory references for Kähler geometry include \cite[Part VI]{cannas_da_silva_lectures_2001} and \cite[Ch. 3]{huybrechts_complex_2005}.

\begin{defn} \label{def_compatible_triple}
	Let $\mfld$ be a manifold. A \emph{pseudo-compatible triple} (or a \emph{pseudo-Kähler triple}) on $\mfld$ is a triple $(\sform, \metric, \cstruct)$, where $\cstruct$ is a complex structure,
	$\metric$ is a pseudo-Riemannian metric, and
	$\sform$ is a symplectic form, all defined on $\mfld$ and satisfying
	\begin{align}
		\metric(X,Y) &= \sform(X,JY), \nonumber
	\end{align}
	for all $X,Y \in \vf(\mfld)$. The triple is said to be \emph{compatible} (or \emph{Kähler}) if $\metric$ is Riemannian.	
\end{defn}

\begin{defn} \label{def_kahler}
	A \emph{pseudo-Kähler manifold} is a manifold $\mfld$ together with a choice of \emph{pseudo-compatible triple} $(\sform, \metric, \cstruct)$. If the triple is compatible, the manifold said to be a \emph{Kähler manifold}.
\end{defn}

A symplectic form $\sform$ belonging to a compatible triple is often also referred to as a \emph{Kähler form}; likewise, a metric $\metric$ belonging to a compatible triple is also referred to as a \emph{Kähler metric}.

Note given two elements of a Kähler triple, the third can be obtained using one of the following expressions (all of which are equivalent)
\begin{align} \label{eq_kahler_compatibility_conditions}
	\metric(X,Y) &= \sform(X,JY), \nonumber \\
	\sform(X,Y) &= \metric(JX,Y), \qquad \forall X,Y \in \vf(\mfld). \\
	\cstruct(X) &= (\contr{X}{\sform})^{\#_\gamma}, \nonumber
\end{align}
As such, we may, by abuse of notation, denote a Kähler triple by two of its elements; we call such a pair a \emph{Kähler pair}. % In what follows, we will be especially interested in parameterizing Kähler structures by their symplectic form and complex structure $(\sform, \cstruct)$. 

\begin{defn}
	Let $(\mfld, \sform, \metric, \cstruct), (\mfldb, \sform', \metric', \cstruct')$ be two Kähler manifolds. A \emph{Kähler manifold isomorphism} is a diffeomorphism $\phi: \mfld \to \mfldb$ which is simultaneously a symplectomorphism and a biholomorphism.
\end{defn}

We say that two Kähler triples $(\sform, \metric, \cstruct), (\sform', \metric', \cstruct')$ on $\mfld$ are \emph{equivalent} if there is a Kähler isomorphism on the manifold considered with the corresponding Kähler structures. Similarly, two Kähler pairs are equivalent if the corresponding Kähler triples are equivalent.


\begin{thm}
Let $\mfld$ be a Kähler manifold and let $p \in \mfld$. Then, there exists an open set $U\subseteq \mfld$ containing $p$ and a function $f \in C^\infty(\mfld)$ such that
\begin{equation*}
	\sform = 2 i \ahd \hd f.
\end{equation*}
Such a function is called a \emph{local Kähler potential on $U$ for $\sform$}.
\end{thm}
See \cite[Theorem 16.5]{cannas_da_silva_lectures_2001} for a proof.

\section{Geometric quantization} \label{sec_gq}

\emph{GQ (Geometric Quantization)} is a procedure which allows us to obtain a quantum theory from a classical system represented by a symplectic manifold (satisfying, as we will see below, some integrality conditions). As the name suggests, it is especially useful as a quantization method that depends on geometric information. The standard reference on geometric quantization is \cite{woodhouse_geometric_1992}. A very nice introduction can be found in \cite[Ch. 22, 23]{hall_quantum_2013}.

We will introduce GQ in two parts: in \secref{sec_gq_without_hf}, we will describe the basic construction, and then we will build upon that in \secref{sec_gq_with_hf} by introducing the so-called half-form correction. \\

\subsection{GQ without half-form correction} \label{sec_gq_without_hf}

The basic setup for GQ starts with a symplectic manifold $(\mfld, \sform)$. In order to define the quantum Hilbert space and the quantization of classical observables in $C^\infty(\mfld; \cmplx)$ (note that we are allowing complex-valued observables), we will need the following:

\begin{defn}
	A Hermitian line bundle with connection $(L, \hip{\cdot}{\cdot}, \covdsymb)$ is a line bundle $\pi: L \to \mfld$ with fibre $\cmplx$ together with a Hermitian structure $\hip{\cdot}{\cdot}: \sect(L) \times \sect(L) \to C^\infty(\mfld; \cmplx)$ defined on its sections  and a compatible connection $\covdsymb$, i.e., a connection such that
	\begin{align*}
		X \cdot \hip{s_1}{s_2} = \hip{\covd{X}s_1}{s_2} + \hip{s_1}{\covd{X}s_2}, \qquad \forall X \in \vf(\mfld).
	\end{align*}
The \emph{curvature 2-form} $\Omega$ of $\covdsymb$ is defined as
	\begin{align*}
		\Omega(X,Y)s = i(\covd{X}\covd{Y} - \covd{Y}\covd{X} - \covd{\lb{X}{Y}})s,
	\end{align*}
	for all $s \in \sect(L)$ and $X,Y \in \vf(\mfld)$.
\end{defn}
If $(L, \hip{\cdot}{\cdot}, \covdsymb)$ is a complex Hermitian line bundle with connection over a symplectic manifold $(\mfld, \sform)$, we define the maps $\inner{\cdot}{\cdot}: \sect(L) \times \sect(L) \to \cmplx \cup \{\infty\}$ and $\norm{\cdot}: \sect(L) \to \reals \cup \{\infty\}$ as
\begin{align} \label{eq_hermitian_integral}
	\inner{s_1}{s_2} &:= \int_\mfld \hip{s_1}{s_2}\frac{\sform^n}{n!},  \\
	\norm{s} &:= \inner{s}{s}^{\frac{1}{2}}, \nonumber
\end{align}
for $s_1, s_2, s \in \sect(L)$. Note that when restricted to $\{s \in \sect(L): \inner{s}{s} < \infty \}$, these define a Hermitian inner product and a norm, respectively.
\begin{rem}\label{rem_local_isometric}
	Given a complex Hermitian line bundle with connection $(L, \hip{\cdot}{\cdot}, \covdsymb)$ on a manifold $\mfld$ and $p \in \mfld$, we can always find an open set $U \subseteq \mfld$ containing $p$ and a local section $s_0$ defined on $U$ such that $\hip{s_0}{s_0} = 1$. We call $s_0$ a \emph{unitary trivialization} (or a \emph{unit section}) of $L$. 
\end{rem}
\begin{defn}
	Let $(\mfld, \sform)$ be a symplectic manifold. A complex Hermitian line bundle with connection $(L, \hip{\cdot}{\cdot}, \covdsymb)$ is said to be a \emph{prequantum line bundle} if it is the case that
	\begin{align*}
		\Omega = i\frac{\sform}{\hbar},
	\end{align*}
	where $\Omega$ is the curvature 2-form of the connection.
\end{defn}


\begin{rem}	
	Given a complex Hermitian line bundle with connection $(L, \inner{\cdot}{\cdot}, \covdsymb)$, $(U, \phi) \in \batlas$ and $s_0 \in \sect(L)|_U$, $s = \psi s_0$ for some $\psi \in C^\infty(U; \cmplx)$. Hence, for $X \in \sect(T\mfld)$, $\covd{X}$ locally takes the form
	\begin{equation*}
		\covd{X} s = \covd{X} (\psi s_0) = \left [(X - \frac{i}{\hbar} \theta_{s_0}(X)) \cdot \psi \right ] s_0,
	\end{equation*}
	where $\theta_{s_0} \in \dform^1(U)$ is such that $\covdsymb s_0 = -\frac{i}{\hbar}  \theta_{s_0} s_0$ and is called the \emph{connection 1-form} associated with $s_0$.
\end{rem}

% \matosc{Pode não ser necessário
% \begin{defn}
% 	Two Hermitian line bundles $(L, \inner{\cdot}{\cdot}, \covdsymb)$ and $(L', \inner{\cdot}{\cdot}', \covdsymb')$ are said to be equivalent if there exists a line bundle isomorphism $\phi: L \to L'$ such that $\phi^*\covdsymb' = \covdsymb$ and $\phi^*\inner{\cdot}{\cdot}' = \inner{\cdot}{\cdot}$
% \end{defn}
% \begin{rem}
% 	Given a prequantization line bundle $(L, \inner{\cdot}{\cdot}, \covdsymb)$ over a symplectic manifold $(\mfld, \sform)$, any equivalent prequantization line bundle can be obtained as $\mathcal{L}' = \mathcal{L} \otimes \mathcal{L}_0$, where $L_0$ is a flat complex Hermitian line bundle. \matosc{especificar o que significa flat?}
% \end{rem}}

Not every symplectic manifold admits a prequantum line bundle. A necessary and sufficient condition for one to exist is that $\left[ \frac{\sform}{2 \pi \hbar}  \right] \in H^2(\mfld, \ints)$, that is, $\left[ \frac{\sform}{2 \pi \hbar} \right]$ represents an integral cohomology class. In other words, the integral of this form over every closed and orientable surface in $\mfld$ is an integer\footnote{This is sometimes called Weil's integrality condition.} (see \cite[Ch. 8.3]{woodhouse_geometric_1992}). This is the content of the following theorem
\begin{thm} \label{thm_quantizable_mfld}
	A symplectic manifold $(\mfld, \sform)$ admits a prequantum line bundle if and only if $\left[ \frac{\sform}{2\pi \hbar} \right]$ is integral, i.e.
	\begin{align*}
		\left[  \frac{\sform}{2\pi\hbar} \right] \in H^2(\mfld, \ints).
	\end{align*}
	In that case, $(\mfld,\sform)$ is said to be \emph{quantizable}.
\end{thm}

\begin{ex}
	The cotangent bundle $(T^*\nfld, \sform)$ of a manifold $\nfld$ with the standard exact symplectic form $\sform = d \spot$ is quantizable.
\end{ex}

A choice of prequantum line bundle allows us to define the prequantum Hilbert space and the prequantization of observables $f \in C^\infty(\mfld; \cmplx)$.
\begin{defn}
A \emph{prequantization} of a symplectic manifold $(\mfld, \sform)$ is a choice of prequantum line bundle $(L, \hip{\cdot}{\cdot}, \covdsymb)$.

The \emph{prequantum Hilbert space} is
\begin{align*}
	\hilb := \close{\left \{s \in \sect(L): \norm{s} < \infty \right \}}^{\norm{\cdot}},
\end{align*}
where $\norm{\cdot}$ is as in \eqref{eq_hermitian_integral}. %That is, the prequantum Hilbert space is the norm closure of the space of smooth sections of $L$, square integrable with respect to the Liouville measure $\frac{\sform^n}{n!}$ on $\mfld$ and the Hermitian structure $\hips$ on the fibers.
For an observable $f \in C^\infty(\mfld;\cmplx)$, the corresponding \emph{prequantum operator} is
\begin{align} \label{eq_prq_def}
	\prq{f} := i \hbar \covd{X_f} + f.
\end{align}
\end{defn}

\begin{prop}
	Let $(L, \hip{\cdot}{\cdot}, \covdsymb)$ be a complex Hermitian line bundle with connection. For $(U, \phi) \in \batlas$, choose a local trivializing section $s_1 \in \sect(L)|_U$ and let $s_2 = e^{-i f / \hbar} s_1$ for some $f \in C^\infty(U; \cmplx)$. Let $\theta_{s_i}, i \in \{1,2\}$ be the corresponding connection 1-forms. Then,   
	\begin{align*}
		\theta_2 = \theta_1 + df
	\end{align*}
\end{prop}

% A change of connection $1$-form as above is called a \emph{gauge transformation}. This reflects the fact that a magnetic field can be represented by the curvature of a Hermitian line bundle and that, in that case, the connection $1$-form is a choice of gauge (see \secref{sec_gq_particle_magnetic}).

For physical reasons, the Hilbert space $\hilb$ associated to a prequantization is too large. We will now deal with this issue. We start by making some remarks regarding vector subbundles of $T\mfld^\cmplx$. First, note that given $\vbund \subseteq T\mfld^\cmplx$, its space of sections is simply
\begin{align*}
	\Gamma(\vbund) = \{X \in \sect(T\mfld^\cmplx) : X(p) \in \vbund(p) \}.
\end{align*}
\begin{defn}
	If $(\mfld, \sform)$ is a $2n$-dimensional symplectic manifold, a \emph{Lagrangian vector subbundle} of $T\mfld^\cmplx$ is a rank-$n$ vector subbundle $\vbund$ of $T\mfld^\cmplx$ such that
\begin{align*}
	\sform(X, Y) = 0, \qquad \forall \, X,Y \in \sect(\vbund).
\end{align*}
\end{defn}
Note that if $E$ is a Lagrangian vector subbundle of $T \mfld^\cmplx$, so is $\overline E$. Indeed, since $\sform$ is real-valued on $\vf(\mfld) \times \vf(\mfld)$ (and is extended by $\cmplx$-linerity to $\vf(\mfld, \cmplx) \times \vf(\mfld, \cmplx)$), then $\sform(X,Y) = \overline{\sform(\overline X, \overline Y)}$ for $X,Y \in \vf(\mfld; \cmplx)$.
\begin{defn} \label{def_polarization}
Let $(\mfld, \sform)$ be a symplectic manifold. A \emph{polarization} $\pol$ of $(\mfld,\sform)$ is a Lagrangian vector subbundle of $T\mfld^\cmplx$ satisfying
\begin{enumerate}
	\item If $X, Y \in \sect(\pol)$, then $\lb{X}{Y} \in \sect(\pol)$
	\item $\dim \left( \pol(p) \cap \overline{\pol(p)} \right)$ is the same for all $p \in \mfld$.
\end{enumerate}
\end{defn}
\noindent Two special types of polarization will play an important role:
\begin{enumerate}
	\item Real polarizations, which satisfy $\pol = \overline{\pol}$.
	\item Complex polarizations, which satisfy $\pol \cap \overline{\pol} = \{0\}$.
\end{enumerate}


\begin{ex} \label{ex_vert_pol}
	Let $\nfld$ be a manifold and consider its cotangent bundle $T^*\nfld$ equipped with the canonical symplectic $2$-form $\sform$. Given $p \in T^*\nfld$, we define
	\begin{align*}
		\pol(p) := \ker d\pi_p,
	\end{align*}
	where $\pi: T^* \nfld \to \nfld$ is the canonical projection. Then $\pol$ is a real polarization called the \emph{vertical polarization}.
\end{ex}
From now on, we will focus on complex polarizations.

Given a polarization $P$ on a symplectic manifold $(\mfld, \sform)$, a choice of prequantization $(L, \hip{\cdot}{\cdot}, \covdsymb)$, and $U \subseteq \mfld$ an open set, a section belonging to the set
\begin{align} \label{eq_pol_sect}
	\sect_\pol(L)|_U := \{s \in \sect(L)|_U: \covd{X} s = 0, \; \forall X \in \Gamma(\overline{\pol})\},
\end{align}
is said to be \emph{polarized} (on $U$). We write $\sect_\pol(L):= \sect_\pol(L)|_\mfld$. Our reduced Hilbert space will then be the norm closure of the set of polarized sections,
\begin{align} \label{eq_reduced_hilb}
	\hilb_\pol := \close{\left \{s \in \sect_\pol(L): \norm{s} < \infty \right \}}^{\norm{\cdot}},
\end{align}
where, again, $\norm{\cdot}$ is as in \eqref{eq_hermitian_integral}.

Given two vector subbundles $R, S \subseteq T\mfld^\cmplx$ such that $R(p) \oplus S(p) = T_p\mfld^\cmplx$, for all $p \in \mfld$, we can uniquely define a bundle isomorphism 
\begin{align} \label{eq_acs_from_subbundle}
	J_{R,S}: T\mfld^\cmplx \to T\mfld^\cmplx
\end{align}
such that
\begin{align}\label{eq_cstruct_from_pol}
	J_{R,S} Z &= i Z, &\forall Z \in \sect(R) \\
	J_{R,S} W &= - i W,  &\forall W \in \sect(S) .\nonumber
\end{align}
Given a complex polarization $\pol$, we define 
\begin{align} \label{eq_cpol_acs}
	J_\pol := J_{\pol, \overline{\pol}}.
\end{align}
Complex polarizations are special in the following sense: 
\begin{prop} \label{prop_cstruct_from_pol}
	Let $\pol$ be a complex polarization on a symplectic manifold $(\mfld, \sform)$. Then 
	% $J_{\pol}$ as in \eqref{eq_cpol_acs} is such that $J_\pol (T\mfld) \subseteq T\mfld$ and $\sform(J_\pol X, J_\pol Y) = \sform(X, Y)$ for all $X, Y \in T\mfld$, i.e. 
	$J_\pol$ defines a compatible complex structure on $(\mfld, \sform)$.
\end{prop}
\begin{proof}
	Let $X \in T \mfld$. Since $\pol \oplus \overline \pol = T \mfld^\cmplx$, then $X = Z + W$ some $Z \in \pol$, $W \in \overline \pol$ and, by \eqref{eq_cstruct_from_pol},
	\begin{align*}
		J_\pol(X) = J_\pol(Z + W) = iZ - iW = \overline{i \overline W - i \overline Z} = \overline{J_\pol(\overline Z+ \overline W)} = \overline{J_\pol(\overline{X})} = \overline{J_\pol(X)}
	\end{align*}
	Hence, $J_\pol(T\mfld) \subseteq T\mfld$. Using point (1) of \defref{def_polarization}, by \propref{prop_nn_ispace} we conclude that $J_\pol$ is a complex structure (in the sense of \defref{def_cstruct}) on $\mfld$. 

	Now, for $X_i = Z_i + W_i$ with $Z_i \in \overline \pol$, $W_i \in \overline \pol$, $i \in \{1,2\}$, we have that
	\begin{align*}
		\sform(J_\pol X_1, J_\pol X_2) &= \sform(i Z_1, i Z_2) + \sform(i Z_1, - i W_2) + \sform(-i W_1, i Z_2) + \sform(-i W_1, -i W_2) \\
		&= \sform(Z_1, W_2) + \sform(W_1, Z_2) \\
		&= \sform(Z_1, Z_2) + \sform(Z_1, W_2) + \sform(W_1, Z_2) + \sform(W_1, W_2) \\
		&= \sform(X_1, X_2).
	\end{align*}
\end{proof}
Hence, a complex polarization on a symplectic manifold $(\mfld, \sform)$ gives it the structure of a pseudo-Kähler manifold. In the opposite direction if $(\mfld, \sform, J)$ is a pseudo-Kähler manifold, then it is easy to check that the $i$ and $-i$ eigenspaces of $J$ define a complex polarization.
\begin{defn} \label{def_kahler_polarization}
	Let $\pol$ be a complex polarization on the symplectic manifold $(\mfld, \sform)$. We say that $\pol$ is a \emph{Kähler polarization} if the bilinear form, defined for $X,Y \in T\mfld$ as
	\begin{align*}
		\metric(X,Y) := \sform(X,J_\pol Y)
	\end{align*}
	is such that $\metric_p$ is positive definite for all $p \in \mfld$, i.e. is such that $(\sform, \metric, J_\pol)$ is a compatible triple on $\mfld$.
\end{defn}


Thus, for $(\mfld, \sform$) a symplectic manifold, the Kähler polarizations are in bijection with the compatible triples having symplectic form $\sform$.

\begin{prop} \label{prop_polarized_holo_sections}
	Let $(L, \hip{\cdot}{\cdot}, \covdsymb)$ be a complex Hermitian line bundle with connection over a Kähler manifold $(\mfld, \sform, \metric, J)$, with corresponding Kähler polarization $\pol$. Let $s_0$ be a unitary section on an open set $U$ as in \remref{rem_local_isometric}. Let $\kpot \in C^\infty(U)$ be a local Kähler potential defined on $U$. Then, the section
	\begin{align*}
		s_\kpot = e^{-\kpot/\hbar} s_0
	\end{align*}
	is such that
	\begin{align*}
		\sect_\pol(L)\big|_U = \{\psi s_\kpot: \psi \in C^\holo(U)\}.
	\end{align*}
\end{prop}
As a result of restricting ourselves to the subspace $\hilb_\pol \subseteq \hilb$ in \eqref{eq_reduced_hilb}, it is no longer possible to prequantize observables $f \in C^\infty(\mfld; \cmplx)$ such that $\prq{f}(\hilb_\pol) \not\subseteq \hilb_\pol$.
\begin{defn}\label{def_observable_quantizable}
	Let $\pol$ be a polarization on the quantizable symplectic manifold $(\mfld, \sform)$. Let $(L, \inner{\cdot}{\cdot}, \covdsymb)$ be a choice of prequantization, and let $f \in C^\infty(\mfld)$. $f$ is said to be \emph{quantizable} with respect to $\pol$ if $\prq{f}(\hilb_\pol) \subseteq \hilb_\pol$.
\end{defn}
We will now state some conditions which guarantee that an observable is quantizable.
\begin{defn}
	Let $\pol$ be a polarization on the symplectic manifold $(\mfld, \sform)$ and let $X \in \vf(\mfld; \cmplx)$. We say that $X$ preserves $\pol$ if
	\begin{align*}
		\forall Y \in \sect(\pol), \quad \lb{X}{Y} \in \sect(\pol).
	\end{align*}
\end{defn}

\begin{thm}\label{thm_obs_quantizability}
	Let $\pol$ be a polarization on a quantizable symplectic manifold $(\mfld, \sform)$ and let $f \in C^\infty(\mfld; \cmplx)$. If $X_f$ preserves $\overline \pol$, then $f$ is quantizable.
\end{thm}

\begin{prop}
	Let $f \in C^\infty(\mfld; \cmplx)$ be such that $df(X) = 0$ for all $X \in \sect(\overline \pol)$. Then $f$ is quantizable and $\prq{f}\psi = f \wf$ for $\psi \in \hilb_\pol$.
\end{prop}

\begin{ex}\label{ex_quantizability_real}
	If $T^*\nfld$ is the cotangent bundle of a manifold $\nfld$ such that $\dim T^*\nfld = 2n$ and $\pol$ is the vertical polarization as in \exref{ex_vert_pol}, then $f \in C^\infty(T^*\nfld)$ is quantizable if and only if $f$ is affine on each fiber, that is, on a cotangent bundle chart $(U, x_1,...,x_n,p_1,...,p_n) \in \batlas$, $f$ can be written as 
	\begin{align*}
		f = f_0(x_1,...,x_n) + \sum_{k=1}^{n} f_k(x_1,...,x_n) p_k,
	\end{align*}
	for some $f_k \in C^\infty(U), k \in \{0,...,n\}$.
\end{ex}

\begin{ex}\label{ex_quantizability_complex}
	Considering the Kähler manifold $(\cmplx^n, \sform, \metric, \cstruct)$ with $\pol$ the corresponding Kähler polarization, $f \in C^\infty(\cmplx^n)$ is quantizable if and only if $f$ is of the form
	\begin{align*}
		f = c_0 + \sum_{j=1}^{n} c_j z^j + \sum_{j=1}^{n} \bar{c}_j \bar{z}^j + \sum_{k,l = 1}^{n}c_{k,l} z^k \bar{z}^l,
	\end{align*}
	for some $c_0, c_j, c_{k,l} \in \cmplx$ with $c_{k,l} = \overline{c_{l,k}}$, $j, k, l \in \{1,...,n\}$.
\end{ex}
The first example can be found found in \cite[Example 23.23]{hall_quantum_2013} and the second in \cite[page 174] {woodhouse_geometric_1992}.
% \begin{ex} \label{ex_pol_from_obs}
% 	Let $(\mfld, \sform)$ be a symplectic manifold of dimension $2n$ and let  $\alpha_1,...,\alpha_n \in \sect(\mfld \times \cmplx)$ be such that 
% 	\begin{align} \label{eq_pol_conditions}
% 		&\pb{\alpha_i}{\alpha_j} = 0, \quad \forall \ i,j \in \{1,...,n\} \nonumber \\
% 		&\lspan\{d\alpha_1,...,d\alpha_n\} = T^*\mfld \\
% 		&X_{\alpha_i} \in T\mfld^\cmplx \setminus T \mfld, \quad \forall \ i \in \{1,...n\}. \nonumber
% 	\end{align}
% 	Then $\{X_{\alpha_i}\}_{i \in\{1,...,n\}}$ defines a complex polarization. 
% \end{ex}

% \begin{defn} \label{defn_lie_series}
% 	Let $\mfld$ be a manifold. For $\tau \in \cmplx$ and $X \in \sect(T\mfld^\cmplx)$, we define the time-$\tau$ exponential of $X$ applied to $f \in C^\infty(\mfld)$ as the formal Lie series
% 	\begin{equation*} % \label{eq_lie_series}
% 	e^{\tau X} \cdot f = \sum_{k=0}^\infty \frac{\tau^k}{k!}X^k(f),
% 	\end{equation*}
% \end{defn}

% Prequantized observables mapping polarized states to polarized states can be directly promoted to fully quantized observables, i.e. $\prq{f} = \q{f}$.
\subsection{GQ with half-form correction} \label{sec_gq_with_hf}

As it stands, geometric quantization does not give us the correct results even in the simplest of cases. 

For a real polarization, a polarized section $s$ generally has infinite norm, which is a consequence of integrating along the leaves of the polarization. For example, if $\mfld = T^*\reals^n$, $\pol$ is the vertical polarization as in \exref{ex_vert_pol} and $L$ is the trivial Hermitian line bundle over $T^*\reals^n$, then $\psi \in \sect_\pol(L)$ is such that $\psi = f(x_1,...,x_n)$ for some $f \in C^\infty(\reals^n; \cmplx)$. Therefore, $\norm{\psi} = \int_{T^*\reals^n}\abs{f(x_1,...,x_n)}^2 \, dx_1...dx_n dp_1...dp_n = \infty$. As a result, $\hilb_\pol = \{0\}$.

In the case of a complex polarization, this is generally not an issue; however, in the end, we still do not obtain physically correct results. Notably, geometric quantization for the harmonic oscillator without half-form correction predicts a spectrum of $\{\hbar \omega n\}_{n \in \nats_0}$ instead of $\{\hbar \omega\left( n+\frac{1}{2} \right)\}_{n \in \nats_0}$, which is known to be the correct one (see e.g. \cite[(2.62)]{griffiths_introduction_2018}), where $\omega$ is the oscillator frequency.

In order to fix these issues, we now introduce the so-called \emph{half-form correction}. We will focus on half-form corrections for complex polarizations since these will be the most relevant ones in the following sections.

\begin{defn}
	Let $\pol$ be a complex polarization on a symplectic manifold $(\mfld, \sform)$. The (complex) \emph{canonical bundle} $\canb_\pol$ of $\pol$ is the complex line bundle whose sections are the differential $n$-forms $\alpha \in \dform^n(\mfld; \cmplx)$ which satisfy
	\begin{align}\label{eq_form_holomorphy}
		\contr{X}{\alpha} = 0
	\end{align}
	for all $X \in \sect(\overline \pol)$.
\end{defn}
\begin{rem} \label{rem_kbund_quotient}
	Considering the complex structure $J_\pol$ as in \propref{prop_cstruct_from_pol}, we see that  condition \eqref{eq_form_holomorphy} implies that the smooth sections of $\canb_\pol$ are precisely the $(n,0)$-forms on $\mfld$.
\end{rem}

\begin{prop} \label{prop_lied_preservation}
	Let $\pol$ be a polarization on the symplectic manifold $(\mfld,\sform)$. Let $X \in \vf(\mfld)$ be a vector field preserving $\pol$ and $\alpha \in \sect(\canb_\pol)$. Then $\lied{X}\alpha \in \sect(\canb_\pol)$, and if $\contr{X} d \alpha = 0$, then $\contr{X} d \left( \lied{X} \alpha \right) = 0$.
\end{prop}
\begin{prop} \label{prop_kbund_pc}
	Let $\pol$ be a complex polarization on the symplectic manifold $(\mfld,\sform)$. For $\alpha \in \sect(\canb_\pol)$ and $X \in \sect(\overline \pol$), the map $\covdsymb : \sect(\overline \pol) \times \sect(\canb_\pol) \to \sect(\canb_\pol)$
	\begin{align}\label{eq_partial_connection}
		\covd{X}\alpha := \contr{X} d \alpha = \lied{X} \alpha
	\end{align}
	is $C(\mfld; \cmplx)$-linear on $\sect(\overline \pol)$, $\cmplx$-linear on $\sect(\canb_\pol)$, and it satisfies the Leibniz rule on $\sect(\canb_\pol)$.
\end{prop}

\begin{proof}
	Since $\alpha \in \sect(\canb_\pol)$, $\contr{X}\alpha = 0$, which implies that
	\begin{align*}
		\lied{X} \alpha = \contr{X} d \alpha = \covd{X}\alpha.
	\end{align*}
	Since $X \in \sect(\pol)$, $X$ preserves $\pol$, which means that by \propref{prop_lied_preservation} $\covd{X}\alpha \in \sect(\canb_\pol)$.

	Since $\lied{X}$ is a derivation operator, it satisfies the Leibniz rule and the condition of $\cmplx$-linearity on $\sect(\canb_\pol)$. As for the condition of $C(\mfld; \cmplx)$-lineariry on $\sect(\overline \pol)$, it comes from the fact that, in this case, $\lied{X} \alpha = \contr{X} d \alpha$ (note that the Lie derivative does not, in general, satisfy $\lied{fX} = f\lied{X}$).
\end{proof}
We call a map as in the previous result a \emph{partial connection} along $\overline \pol$. We define polarized sections similarly to the case of connections:
\begin{defn}
	Let $\pol$ be a polarization on a symplectic manifold $(\mfld, \sform)$ and let $\covdsymb$ be a partial connection along $\overline \pol$ on a complex line bundle $L$ on $\mfld$. Let $U \subseteq \mfld$ be an open set. A section $\alpha \in \sect(L)|_U$ is said to be \emph{polarized} (on $U$) if
	\begin{align} \label{eq_cbund_psec}
		\covd{X} \alpha = 0
	\end{align}
	for all $X \in \sect(\overline \pol)|_U$. We denote the set of such sections as $\sect_\pol(L)|_U$ and define $\sect_\pol(L) := \sect_\pol(L)|_\mfld$.
\end{defn}

\begin{rem}
	Again, considering the complex structure $\cstruct_\pol$ on $\mfld$ coming from a complex polarization $\pol$ as in \propref{prop_cstruct_from_pol}, we note that the space of polarized sections $\sect_\pol(\canb_\pol)$ can be identified with the set of $\cstruct_\pol$-holomorphic $(n,0)$-forms on $\mfld$, $\dform^{n, \holo}(\mfld)$.
\end{rem}

Note that by \propref{prop_lied_preservation}, for all $X \in \sect(\overline \pol)$, if $\alpha \in \sect_\pol(\kbund_\pol)$, then $\covd{X}\alpha \in \sect_\pol(\kbund_\pol)$.

% \begin{ex}
% 	If $\mfld = T^*\reals^n$ and $\pol$ is the vertical polarization on $\mfld$ (\exref{ex_vert_pol}), then $\alpha \in \sect(\canb_\pol)$ if and only if
% 	\begin{align*}
% 		\alpha = f(x,p) dx_1\wedge...\wedge dx_n
% 	\end{align*}
% 	for $f \in C^\infty(T^*\reals^n)$, and $\alpha$ is polarized if and only if
% 	\begin{align*}
% 		\alpha = g(x) dx_1 \wedge ... \wedge dx_n
% 	\end{align*}
% 	for $g \in C^\infty(\reals)$.
% \end{ex}

% \begin{prop} \label{prop_canb_form_lift}
% 	If $\leaves_\pol$ \matosc{definir} is a smooth manifold and $\alpha \in \sect(\canb_\pol)$ is polarized, then $\exists ! \tilde \beta \in \dform^n(\leaves)$ satisfying
% 	\begin{align*}
% 		\alpha = \pi^*(\beta)
% 	\end{align*}
% 	where $\pi: \mfld \to \leaves$ is the quotient map. On the other hand, if $\beta \in \dform^n(\leaves)$, then $\pi^* \beta$ is a polarized section of $\canb_\pol$.
% \end{prop}
% \begin{prop}
% 	If $\leaves_\pol$ is a smooth manifold and $X \in \vf(\mfld)$ preserves $\pol$, then $\exists ! Y \in \leaves_\pol$ satisfying
% 	\begin{align*}
% 		\pi_*(X) = Y
% 	\end{align*}
% 	Moreover, if $\alpha \in \sect(\canb_\pol)$ is such that $\alpha = \pi^* (\beta)$ (cf. \propref{prop_canb_form_lift}), then
% 	\begin{align*}
% 		\lied{X}(\pi^*(\beta)) = \pi^*(\lied{Y}\beta)
% 	\end{align*}
% \end{prop}




% From here on we will assume that $\leaves_\pol$ is an orientable manifold with volume form $\beta \in \dform^n(\leaves_\pol)$.

% \begin{defn}
% 	Fix a volume form $\beta \in \dform^n(\leaves_\pol)$. A section $\eta \in \canb$ is non-negative (relative to $\beta$) if
% 	\begin{align*}
% 		\eta = k \pi^* \beta, \qquad \text{for } k \in \reals^+
% 	\end{align*}
% \end{defn}

% Note that when $\leaves_\pol$ is orientable, $\canb_\pol$ is trivializable since $\pi^* \beta$ is a non-vanishing section.
\begin{defn}
	A \emph{square root of $\canb_\pol$} is a complex line bundle $\sqrtb_\pol$ over $\mfld$ along with an isomorphism $\phi: \sqrtb_\pol \otimes \sqrtb_\pol \to \canb_\pol$. 
\end{defn}

\begin{prop}
	Let $\pol$ be a polarization on a symplectic manifold $(\mfld, \sform)$ and let $\covdsymb$ be the partial connection on $\kbund_\pol$ as in \propref{prop_kbund_pc}. Let $\sqrtb_\pol$ be a square root of $\canb_\pol$. Then there is a unique linear operator $\covd{X}:\sect(\sqrtb_\pol) \to \sect(\sqrtb_\pol)$ satisfying
	\begin{align*}
		\covd{X}(fs_1) &= (X \cdot f) s_1 + f \covd{X} s_1, \\
		\covd{X}(s_1\otimes s_2) &= (\covd{X}s_1)\otimes s_2 + s_1 \otimes (\covd{X} s_2), 
	\end{align*}
	and if $X$ preserves $\bar \pol$, then there is an unique linear operator $\lied{X}:\sect(\sqrtb_\pol) \to \sect(\sqrtb_\pol)$ satisfying 
	\begin{align*}
		\lied{X}(fs_1) &= (X \cdot f) s_1 + f \lied{X} s_1, \\
		\lied{X}(s_1\otimes s_2) &= (\lied{X}s_1)\otimes s_2 + s_1 \otimes (\lied{X} s_2),
	\end{align*}
	for all $X \in \sect(\overline \pol), f \in C^\infty(\mfld;\cmplx)$ and $s_1, s_2 \in \sect(\sqrtb_\pol)$.
	$\covdsymb:\sect(\overline \pol)\times\sect(\sqrtb_\pol) \to \sect(\sqrtb_\pol)$ is a partial connection on $\sqrtb_\pol$.
\end{prop}
We then say that $s \in \sect(\sqrtb_\pol)$ is polarized if, for all $X \in \sect(\overline\pol)$, $\covd{X} s = 0$, where $\covdsymb$ is as in the previous proposition.

We can now define a partial connection on the tensor bundle $L \otimes \sqrtb_\pol$ as follows

\begin{prop}
	Let $\pol$ be a polarization on a symplectic manifold $(\mfld,\sform)$ with prequantization $(L, \hip{\cdot}{\cdot}, \covdsymb)$. Let $\sqrtb_\pol$ be a square root of $\canb_\pol$. Write $s \in \sect(L\otimes \sqrtb_\pol)$ locally as $s = \tilde s \otimes \nu$. Then the tensor product partial connection $\covdsymb : \sect(\overline \pol) \times \sect(L \otimes \sqrtb_\pol) \to \sect(L \otimes \sqrtb_\pol)$, written as
	\begin{align} \label{eq_tensor_connection}
		\covd{X} s= \left (\covd{X} \tilde s \right ) \otimes \nu +  \tilde s \otimes \left (\covd{X}\nu \right ),
	\end{align}
	does not depend on the choice of $\tilde s,\nu$, and is thus defined globally.
\end{prop}

If $\eta, \nu \in \dform^n(\mfld)$, then $\exists ! f \in C^\infty(\mfld)$ such that $\eta = f \nu$. To simplify the notation in the following proposition, we denote such an $f$ by $\frac{\eta}{\nu} := f$.
\begin{prop}\label{prop_sqrtb_hermitian_struct}
	Let $(\mfld, \sform)$ be a $2n$-dimensional symplectic manifold, let $\pol$ be a complex polarization and let $\delta_\pol$ be a square root of $\kbund_\pol$. Then there is a unique Hermitian structure $\hip{\cdot}{\cdot}: \sect(\delta_\pol) \times \sect(\delta_\pol) \to C^\infty(\mfld)$ on $\delta_\pol$ such that, for $s \in \sect(\sqrtb_\pol)$,
	\begin{align*}
		\hip{s}{s} = \left( \frac{(-1)^{\frac{n(n-1)}{2}}(-i)^n}{2^n} \frac{\overline{(s \otimes s)}\wedge(s\otimes s)}{\sform^n/n!} \right)^\frac{1}{2}.
	\end{align*}
\end{prop}
% \begin{proof}
% 	\matosc{Exercise in book, need to complete}
% \end{proof}
Since $L$ is a Hermitian line bundle, its Hermitian structure plus that on $\sqrtb_\pol$ given in  \propref{prop_sqrtb_hermitian_struct} give rise to a Hermitian structure on $L \otimes \sqrtb_\pol$. This is the last element we need in order to define the Hilbert space for GQ with half-form correction. Define first the space of unpolarized sections as 
\begin{align}\label{eq_hilb_whf}
	\hilb^\text{hf} := \close{\{s \in \sect(\lbund \otimes \sqrtb): \norm{s} < \infty \}}^{\norm{\cdot}},
\end{align}
where, again, $\norm{\cdot}$ is as in \eqref{eq_hermitian_integral}.
%where $\inner{s_1}{s_2}_{\text{hf}} := \int_\mfld \hip{s_1}{s_2}$ and $\norm{s}^2_\text{hf} := \int_\mfld \hip{s}{s}$, for $s_1, s_2, s \in \sect(L \otimes \sqrtb)$.
\begin{defn}
	Let $\pol$ be a complex polarization on a quantizable symplectic manifold $(\mfld, \sform)$ and let $(L, \hip{\cdot}{\cdot}, \covdsymb)$ be a prequantization of this manifold. Let $\sqrtb_\pol$ be a square root of the canonical bundle $\canb_\pol$.  Then, the \emph{half-form corrected Hilbert space} for a complex polarization $\pol$ on $\mfld$ is
	\begin{align} \label{eq_reduced_hilb_whf}
		\hilb^\text{hf}_\pol := \close{\{\psi \in \hilb^\text{hf}: \covd{X} s = 0, \; \forall X \in \Gamma(\overline{\pol})\}}^{\norm{\cdot}},
	\end{align}
	where $\hilb$ is as in \eqref{eq_hilb_whf}.
\end{defn}
% By an abuse of notation, we use the same notation for \eqref{eq_reduced_hilb} and \eqref{eq_reduced_hilb_whf}. It will be obvious from context whether we are using GQ with or without half-form correction.

We say that $f \in C^\infty(\mfld; \cmplx)$ is \emph{quantizable with half form correction} whenever $X_f$ preserves $\overline \pol$. Note that the observables which are quantizable with half-form correction are, in general, a subset of those which are quantizable without half-form correction (see \thmref{thm_obs_quantizability}).
\begin{prop}
	Let $\pol$ be a complex polarization on a quantizable symplectic manifold $(\mfld, \sform)$ and let $(L, \hip{\cdot}{\cdot}, \covdsymb)$ be a prequantization of this manifold. Let $\sqrtb_\pol$ be a square root of the canonical bundle $\canb_\pol$. Let $f \in C^\infty(\mfld; \cmplx)$ be such that $X_f$ preserves $\overline \pol$. Define the \emph{prequantization of} $f$ \emph{with half-form correction} as
	\begin{align} \label{eq_prq_whf}
		\prqhf{f} s := (\prq{f} \tilde s)\otimes \nu + i \hbar \tilde s \otimes \lied{X_f}\nu,
	\end{align}
	where $\tilde s$ is locally written as $ s = \tilde s \otimes \nu$, with $\tilde s$ being a section of $L$ and $\nu$ a section of $\sqrtb_P$. Then $\prqhf{f} s$ does not depend on the choice of $\tilde s$ and $\nu$.
	%  and, for $g \in C^\infty(\mfld)$ preserving $\overline P$, satisfies
	% \begin{align*}
	% 	\frac{1}{i\hbar}\lb{\q{f}}{\q{g}}=\q{\pb{f}{g}}
	% \end{align*}
	% on $\sect_\pol(L\otimes\sqrtb_\pol^\cmplx)$.
\end{prop}
\begin{ex}
	Consider the symplectic manifold $T^*\reals$ with complex polarization given by the complex structure induced by the coordinate $z = x-\frac{ip}{m \zeta}$ for some $\zeta \in \reals^+$ and trivial Hermitian line bundle with trivial connection. Choose $\sqrtb_\pol$ to be trivial with trivializing section $\sqrt{dz}$, where $\sqrt{dz}$ is such that $\sqrt{dz} \otimes \sqrt{dz} = dz$. Let
	\begin{align*}
		H(x,p) = \frac{p^2 + (m \zeta x)^2}{2m}
	\end{align*}
	be the harmonic oscillator Hamiltonian, where $\zeta$ is the frequency. Then $X_H$ preserves $\overline \pol$ and the spectrum of $\prqhf{H}$ is
	\begin{align*}
		\left \{\hbar \zeta \left( n + \frac{1}{2} \right) \right \}_{n \in \nats_0}.
	\end{align*}
\end{ex}

\begin{rem} \label{rem_full_quantization}
	Let $\pol$ be a polarization on a prequantizable symplectic manifold $(\mfld, \sform)$, along with a choice of prequantization and square root bundle $\sqrtb$. If an observable $f \in C^\infty(\mfld)$ is quantizable with half-form correction, we define its \emph{quantization} as 
	\begin{align*}
		\q{f} := \prqhf{f}.
	\end{align*}
	However, as \exref{ex_quantizability_real} and \exref{ex_quantizability_complex} illustrate, there are severe restrictions on the observables that are quantizable. In general, one must deal with the problem of observables not preserving the polarization on a case-by-case basis. We will employ the following solution. If $g \in C^\infty(\mfld; \cmplx)$ is an observable not preserving $\pol$, but $g = g(f_1,...,f_n)$ for some $f_1,...,f_n \in C^\infty(\mfld; \cmplx), n \in \nats$, all of which preserve the polarization, then assuming $\prq{f_1},...,\prq{f_n}$ commute, we take 
	\begin{align} \label{eq_quant_nopol}
		\q{g} := g(\prq{f_1},...,\prq{f_n}).
	\end{align}
\end{rem}

% \begin{rem} \label{rem_full_quantization_whf}
% 	As in \remref{rem_full_quantization}, if an observable $f \in C^\infty(\mfld)$ is quantizable, we define its quantization as
% 	\begin{align*}
% 		\q{f} := \prqhf{f}
% 	\end{align*} 	
% 	However, the restrictions evidenced by \exref{ex_quantizability_real} and \exref{ex_quantizability_complex} also apply to the case with half-form correction. We will employ the same solution as in the case without half-form correction. If $g \in C^\holo(\mfld)$ is an observable not preserving $\pol$, but $g = g(f)$ for some $f \in C^\holo(\mfld)$ which preserves the polarization, then we take 
% 	\begin{align} \label{eq_quant_nopol}
% 		\q{g} := g(\prqhf{f})
% 	\end{align}
% \end{rem}

\section{Complex flows and geodesics in the space of Kähler metrics} \label{sec_complex_flows}

\subsection{Complexification of real analytic flows} 
Here, we summarize part of the theory developed in \cite{mourao_complexified_2015}, which will be essential in order to change the geometry of the surfaces in \secref{sec_geometry_dependence}.

\begin{defn} \label{def_general_lie_series}
	Let $A: V \to V$ be a function on a vector space $V$ over a field $\field$ and let $v \in V$, $\tau \in \field$. We define the \emph{formal time-$\tau$ exponential series of $A$ applied to $v$} as 
	\begin{align} \label{eq_general_lie_series}
		e^{\tau A}\cdot v := \sum_{k=0}^{\infty}\frac{(\tau A)^k(v)}{k!}.
	\end{align}
\end{defn}

\begin{defn}\label{def_lie_series}
	For $\tau \in \cmplx$ and $X \in \vf^\ra(\mfld)$, we define the time-$\tau$ exponential of $X$ applied to $f \in C^\ra(\mfld)$ as the formal Lie series given by $e^{\tau X} \cdot f$ (as in \defref{def_general_lie_series}).
\end{defn}

The following lemma, adapted from \cite{groebner_general_1967}[p. 10, Theorem 3] and stated in \cite{mourao_complexified_2015}, ensures the convergence of this series for small $\abs{\tau}$:
\begin{lem} \label{lem_lie_series}
Let $\mfld$ be a compact real analytic manifold and $X$ be a real analytic vector field on $M$. For each $f \in C^\ra(\mfld)$, there exists $T_f \in \reals^+$ such that, for $\tau \in D_{T_f} = \{\tau \in \cmplx: \abs{\tau} < T_f\}$, the Lie series
\[
e^{\tau X} \cdot f = \sum_{k=0}^\infty \frac{\tau^k}{k!}X^k(f)
\]
converges uniformly on compact subsets of $\mfld$ and defines a real analytic function on $\mfld \times D_{T_f}$.
\end{lem} 

\begin{rem}
	As a result of the Commutation Theorem (see \cite[p.17, Theorem 6]{groebner_general_1967}), if $t \in \reals$, $X \in \vf^\ra(\mfld)$, and $f \in C^\ra(\mfld; \cmplx)$, then $(e^{t X} \cdot f)  (p)= ((\phi^X_t)^* f) (p)$.
\end{rem}

\begin{thm}[\hspace{1sp}\cite{mourao_complexified_2015}, Theorem 2.5] \label{thm_coordinate_evolution}
Let $\mfld$ be a compact complex manifold with complex structure $J_0$ and $X \in \vf^\ra(\mfld)$. Let $(U, z_1,...,z_n)$ be a set of $J_0$-holomorphic coordinates containing $p \in \mfld$. Then, there exists a $T \in \reals^+$ such that, for every $\tau \in D_T = \{\tau \in \cmplx : \abs{\tau} < T\}$, the functions
\[z_j^\tau = e^{\tau X} \cdot z_j, \qquad j = 1,...,n,\]
define a new complex structure $J_\tau$ on some open neighborhood $V \subset U$ of $p$ and form a local set of $J_\tau$-holomorphic coordinates on that neighborhood.
\end{thm}

% Finally, the next theorem states that there is a global complex structure under which the evolved coordinates are holomorphic and establishes the existence of a biholomorphism that represents the flow locally:

\begin{thm}[\hspace{1sp}\cite{mourao_complexified_2015}, Theorem 2.6] \label{thm_complex_flow}
Let $\mfld$ be a compact complex manifold with complex structure $J_0$ and $X\in\vf^\ra(\mfld)$. Then, there exists a $T \in \reals^+$ such that, for $\tau \in D_T=\{\tau \in \cmplx : \abs{\tau} < T\}$, there exists a global complex structure $\cstruct_\tau$ on $\mfld$ extending the complex structure given in local $\cstruct_\tau$-holomorphic charts as in \thmref{thm_coordinate_evolution}, along with a unique biholomorphism
\[
\phi_\tau : (\mfld, \cstruct_\tau) \to (\mfld, J_0),
\]
which, on local $\cstruct_0$-holomorphic coordinates, acts as $e^{\tau X}$. This biholomorphism will be referred to as the \emph{complex-time flow} of X.
\end{thm}

\begin{rem}
	Even though it is usually referred to as a flow, $\phi_\tau$ is not a flow in the sense that it is not true, in general, that $\phi_{\tau + \sigma} = \phi_\tau \circ \phi_\sigma$, assuming $\phi_{\tau + \sigma}, \phi_{\tau}, \phi_{\sigma}$ are defined for $\tau, \sigma \in \cmplx$. This is due to the dependence of the map $\phi_\tau$ on the starting complex structure. Writing this dependency explicitely, we obtain the following commutative diagram
	\begin{center}
		\begin{tikzcd}[column sep=small]
			& (\mfld, J_{\tau + \sigma}) \arrow[dd, "\phi_{\sigma + \tau}^{J_0}"] \ar[dl, "\phi_{\sigma}^{J_\tau}"] \ar[dr, "\phi_{\tau}^{J_\sigma}"] &	\\
			(\mfld, J_{\tau}) \arrow[dr, "\phi_{\tau}^{J_0}"] &	& (\mfld, J_{\sigma}) \arrow[dl, "\phi_{\sigma}^{J_0}"]	\\
			&	(\mfld, J_0)	&
		\end{tikzcd}
	\end{center}		
\end{rem}


\begin{thm}[\hspace{1sp}\cite{mourao_complexified_2015}, Theorem 3.1]
	Let $U \subseteq \cmplx^n$ be an open set and consider $(U, \sform, J_0)$ with the induced complex structure, a real analytic symplectic form $\sform \in \dform^{2,\ra}(\mfld)$ and $H \in C^\ra(U)$. Take $f \in C^\ra(\mfld)$ and $V \subseteq U$ an open set such that $e^{\tau X_H} \cdot f$ converges uniformly on compact subsets of $V$. Then it is the case that
	\begin{align*}
		e^{\tau \lied{X_H}} \cdot X_f = X_{e^{\tau X_H}\cdot f},
	\end{align*}
	where $e^{\tau \lied{X_H}} \cdot X_f$ is as in \defref{def_general_lie_series}.
\end{thm}

\begin{thm}[\hspace{1sp}\cite{mourao_complexified_2015}, Theorem 4.1] \label{thm_kpot_evolution}
	Let $\mfld$ be a compact Kähler manifold with compatible triple $(\sform, \cstruct_0, \metric_0)$. Assume that every element of this triple is analytic. Let $H \in C^\ra(\mfld)$. Then there exists a $T \in \reals^+$ such that, for all $\tau \in D_T = \{\tau \in \cmplx : \abs{\tau} < T\}$, $(\sform, \cstruct_\tau)$ is a Kähler pair, where $J_\tau = \phi_\tau^* J_0$ and $\phi_\tau$ is the complex-time flow of $X_H \in \vf^\ra(\mfld)$ as in \thmref{thm_complex_flow}.
\end{thm}

\begin{thm}[\hspace{1sp}\cite{mourao_complexified_2015}, Theorem 4.4] \label{thm_polarization_evolution}
	Under the conditions of \thmref{thm_kpot_evolution}, the Kähler polarization associated to $J_\tau$ is given by
	\begin{align} \label{eq_evolved_pol}
		\pol_\tau = (\phi_\tau^{-1})_*\pol_0 = e^{\tau \lied{X_H}}\pol_0,
	\end{align}
	where the right hand side of \eqref{eq_evolved_pol} is to be interpreted as the polarization obtained by applying the operator $e^{\tau \lied{X_H}}$ to a the Hamiltonian vector fields generated by a local set $J_0$-holomorphic coordinates $(U,z_1,...,z_n)$.
\end{thm}

\begin{rem}\label{rem_geometry_evolution}
	One crucial aspect is that, although $\phi_\tau$ as in \thmref{thm_complex_flow} is a biholomorphism by construction, it is not necessarily a symplectomorphism. This implies that the Kähler pairs $(\sform, J_0)$ and $(\sform, J_\tau)$ will, in general, be nonequivalent. We will exploit this in \secref{sec_geometry_dependence} in order to change the Kähler structure of a surface.
\end{rem}


Before we move on, we emphasize that most of the theory presented here holds for compact $\mfld$ and relies on there being a set $D_T=\{\tau \in \cmplx : \abs{\tau} < T\}$ for some $T \in \reals^+$ where it is valid. For concrete (especially non-compact) examples, such as those considered in \secref{sec_geometry_dependence}, this existence result does not give us explicit results, and the convergence conditions have to be checked on a case-by-case basis. \\

% Simplified example:
% 	\begin{itemize}
% 		\item $H = \frac{p^2}{2}$, $X_{H} = p \pd{q}$
% 		\item $\phi_t^{X_H}(q,p) = (q+tp,p)$
% 	\end{itemize}    
% 	\begin{align*}
% 		\left (\phi_t^{X_H} \right )^*(z)\big|_{t=is} &= \left (\phi_t^{X_H} \right )^*(q+ip)\big|_{t=is} \\
% 		&= (q+tp+ip)\big|_{t=is} = q+i(s+1)p \\
% 		&=: \left (\phi_{is}^{X_H} \right )(q,p)
% 		\end{align*}
% 	\begin{itemize}
% 		\item $\phi_{is}^{X_H}(x,p) = (x,(s+1)p)$
% 	\end{itemize}

\subsection{Geodesics in the space of Kähler metrics} \label{sec_geodesics}

In this section, we will see that the change in the Kähler structure of a manifold induced by the complex-time flow of a Hamiltonian vector field, as in \remref{rem_geometry_evolution}, can also be seen as a geodesic in a space of Kähler structures on the manifold (to be defined precisely below). We are very brief in this section; for a more detailed exposition on these aspects, see e.g. \cite{donaldson_symmetric_1999}.

\begin{defn}
Given a differentiable manifold $\mfld$, we denote by 	$\kset(\mfld)$ the set of all compatible Kähler triples on $\mfld$.	
\end{defn}

As stated in \secref{sec_kahler}, by abuse of notation, we will treat Kähler triples intechangeably with Kähler pairs. As such, in what follows, we will omit the metric, writing $(\sform,\cstruct) \in \kset(\mfld)$.

Given a symplectic form $\sform \in \dform^2(\mfld)$ on a manifold $\mfld$, we will consider the subset $\fcset(\sform, \cstruct) \subseteq \kset(\mfld)$ of compatible Kähler triples having complex structure $\cstruct$ and whose Kähler form lies in the same cohomology class as $\sform$, that is
\begin{align*}
	\fcset(\mfld, \cstruct) = \{\alpha : (\alpha,J) \in \kset(\mfld), \, [\alpha] = [\sform]\}.
\end{align*}
Assuming that $\mfld$ is compact, by the $\hd \ahd$-lemma \cite[Corollary 3.2.10]{huybrechts_complex_2005}, cohomologous Kähler forms can be related via a global potential. Defining the set 
\begin{align*}
	\fcpset(\sform, J) = \{u \in C^\infty(\mfld,\reals): (\sform + i \hd \ahd u, J) \in \kset(\mfld) \} 
\end{align*}
and defining, for $u \in C^\infty(\mfld)$,
\begin{align}\label{eq_distorted_form}
	\sform_u =  \sform + i \hd \ahd u,
\end{align}
we can write $\fcset(\sform, \cstruct)$ as
\begin{align*}
	\fcset(\sform, \cstruct) &\cong \fcpset(\sform, \cstruct) / \reals,
\end{align*}
where we take the quotient by the real constant functions on $\mfld$, since potentials differing by a constant define the same form and, thus, $\sform_u = \sform_{u+c}$, $c \in \reals$.
\begin{rem}
	Since $C^\infty(\mfld)$ is a Fréchet vector space (see e.g. \cite[Definition 2.4]{conway_course_1990}), it is an infinite-dimensional manifold, and since $\fcpset(\sform, \cstruct)$ is an open subset of $C^\infty(\mfld)$, it is a Fréchet manifold modeled on the same space. That $\fcpset(\sform, \cstruct)$ is an open subset of $C^\infty(\mfld)$ can be seen by noting that it is defined by an open condition; indeed, given a local set of $J$-holomorphic coordinates $(U, z_1,...,z_n)$,
\begin{align*}
	u \in \fcpset(\sform, J) \iff \left( \frac{\partial^2 (\kpot + \hat u)}{\partial z_j \partial \bar z_k} \right)_{j,k \in \{1,...,n\}} \text{ is positive definite},
\end{align*}
where $\kpot$ is a local Kähler potential for $\sform$ and $\hat u$ is a local representative of $u$. We can thus perform the identification $T_u \fcpset(\sform, \cstruct) \cong C^\infty(\mfld)$.
\end{rem}

\begin{defn}
	Given $F,G \in T_u \fcpset \cong C^\infty(\mfld)$, the \emph{Mabuchi metric} is defined as
	\begin{align*}
		\inner{F}{G}_u := \int_\mfld FG \, \frac{\sform_u^n}{n!},
	\end{align*}
	where $\sform_u$ is as in \eqref{eq_distorted_form}.
\end{defn}

The corresponding geodesic equation is as follows (see \cite[Eq. (12)]{donaldson_symmetric_1999}).

\begin{prop}
	The geodesic equation corresponding to the Mabuchi metric is
	\begin{align*}
		\ddot u_t = \frac{1}{2}\norm{\grad^{u_t}\dot u_t}^2_{u_t},
	\end{align*}
	where $\grad^{u_t}$ and $\norm{\cdot}_{u_t}$ are the usual gradient and norm, calculated using the Riemannian metric $g_{u_t}$ on $\mfld$ corresponding to the Kähler pair $(\sform_{u_t},\cstruct)$.
\end{prop}

\begin{rem} \label{rem_symplectic_picture}
	Going back to \remref{rem_geometry_evolution}, let $\mfld$ be a compact Kähler manifold with Kähler structure defined by the Kähler pair $(\sform, J_0)$, and let $\phi_\tau$ be the complex-time flow of the Hamiltonian vector field $X_H \in \vf^\ra(\mfld)$ for some $H \in C^\ra(\mfld)$. By \thmref{thm_complex_flow} and \thmref{thm_kpot_evolution}, we assume that $\phi_\tau$ is defined for $\tau \in D_T, T \in \reals^+$, and that $(\sform, \cstruct_\tau)$ is a Kähler pair, where $J_\tau$ is the induced complex structure. Then it is general not the case that $(\sform, J_\tau)$ is equivalent to $(\sform, J_0)$. This means that we obtain a nontrivial path in $\kset(\mfld)$. 
	
	Here, we are assuming that $\sform$ is fixed while $J_\tau$ varies; this is the so-called the \emph{symplectic picture}. The theory developed in this section, however, assumed the complex structure fixed while the symplectic form was allowed to vary; this is the so-called \emph{complex picture}. It turns out that both formulations are equivalent in the sense that, defining $\sform_\tau := \left( \phi_\tau^{-1} \right)^* \sform$, $(\sform, J_\tau)$ and $(\sform_\tau, J)$ are equivalent.
\end{rem}


The main theorem of this section is 
\begin{thm}[Theorem 9.1, \cite{mourao_complexified_2015}] \label{thm_complex_flow_geodesics}
	Let $\mfld$ be a compact manifold and let $(\sform, \cstruct_0) \in \kset(\mfld)$. Let $H \in C^\ra(\mfld)$ and $X_H \in \vf^\ra(\mfld)$ be an analytic and real Hamiltonian vector field. Consider its complex flow $\phi_\tau$ such that it is defined for $\tau \in D_T=\{\tau \in \cmplx : \abs{\tau} < T\}$, $T \in \reals^+$ . Let $\sform_\tau$ be as in \remref{rem_symplectic_picture} and consider all such forms for $\tau = it, 0 \leq t < T$. Then $\{\sform_{it}\}_{t \in [0,T[}$ corresponds to a geodesic path in $\fcpset(\sform, J_0)$.
\end{thm}

\section{Kähler toric manifolds and Guillemin-Abreu theory} \label{sec_coordinates}

	Here we will introduce the action-angle and toric holomorphic coordinate formalisms following mainly \cite{abreu_kahler_2003}. This will be later used in \secref{sec_geometry_dependence}, where we perform computations which become much simpler when using action-angle coordinates.

	\begin{defn} \label{def_ktoric_action}
		A \emph{Kähler toric manifold} is a closed, connected $2n$-dimensional Kähler manifold $(\mfld,\sform,\cstruct)$ with a Hamiltonian toric action
		\begin{align*}
			\tact:\torus^n\to\diff(\mfld,\sform,\cstruct),
		\end{align*}
		which is both effective and holomorphic with respect to the complex structure $J$.
	\end{defn}

	Throughout this section, $(\mfld,\sform,\cstruct,\tact)$, will denote a Kähler toric manifold as above. Since $\tact$ is Hamiltonian, associated to it there is a moment map, which we will denote by $\mmap: \mfld \to \reals^n$. 
	\begin{defn}
		Let $P \subseteq \reals^n$ be a convex polytope. Then $P$ is said to be \emph{Delzant} if 
		\begin{enumerate}
			\item There are exactly $n$ edges meeting at each vertex.
			\item Each edge is of the form $p_i+tv_i$, with $t \in \reals^+_0$, $v_i \in \ints^n$ and $p_i \in \reals^n$, $i \in \{1,...,n\}$.
			\item The $v_i$ in $(2)$ form a basis of $\ints^n$.
		\end{enumerate}
	\end{defn}
	It is the case that (see \cite{delzant_hamiltoniens_1988}) $P := \mmap(\mfld)$ is a Delzant polytope and that
	\begin{align} \label{eq_open_free_set}
		\mfld^\circ := \mmap^{-1}(\poly^\circ) = \{p \in M: \tact \text{ is free at }p\},
	\end{align}
	where $P^\circ$ is the interior of the Delzant polytope. This is the open set where both sets of coordinates, action-angle and toric holomorphic, will be defined.

	% \begin{thm}
	% 	Let $(\mfld, \sform, \tact)$ be a compact and connected 
	% \end{thm}

	% \begin{thm}
	% 	Let $(\mfld, \sform, \cstruct, \tact)$ be a Kähler toric manifold with moment polytope $\poly \subset \reals^n$. Then
	% 	\begin{enumerate}
	% 		\item $(\mfld, \sform, \tact)$ is equivariantly symplectomorphic to $(\mfld_\poly, \sform_\poly, \tact_\poly)$.
	% 		\item $(\mfld, \cstruct, \tact)$ is equivariantly biholomorphic to $(\mfld_\poly, \cstruct_\poly, \tact_\poly)$.
	% 	\end{enumerate}
	% \end{thm}
	

	% \begin{defn}
	% 	Let $(\mfld,\sform,\cstruct,\hact)$ be a Kähler toric manifold and let $\mfld^\circ$ be as in \eqref{eq_open_free_set}. Then a parametrization 
	% 	\begin{align*}
	% 		\phi: 
	% 	\end{align*}
	% 	of $\mfld^\circ$ is said to be \emph{toric holomorphic} if
	% \end{defn}

	We start by describing toric holomorphic coordinates. It turns out that such a set of coordinates always exist (see \cite[Appendix A]{abreu_kahler_2003}),
	\begin{align} \label{eq_kt_coords}
		\mfld^\circ \cong \cmplx^n/2\pi i \ints^n = \reals^n\times i \torus^n = \{v+i\theta:u\in \reals^n, v \in \reals^n/\ints^n\},
	\end{align}
	and that, in these coordinates, the action $\tact$ is such that
	\begin{align*}
		\tact(t) (v+i\theta)=v+i(\theta+t), \qquad t \in \torus^n.
	\end{align*}
	In these coordinates the complex structure is simply
	\begin{align*}
		\tilde{\cstruct} = \begin{bmatrix}
			0 & \vdots & - I \\
			\cdots & \cdots & \cdots \\
			I & \vdots & 0
		\end{bmatrix},
	\end{align*}
	i.e. it is the multiplication by the imaginary unit $i$, where $I$ represents the $n\times n$ identity matrix and $\tilde \cstruct$ is the local matrix representation of $\cstruct$ in toric holomorphic coordinates.
	
	There exists a $\torus^n$-invariant potential $\kpot \in C^\infty(\mfld^\circ)$ such that $\sform = 2i\hd \ahd \kpot$. Since $\sform$ is $\tact$-invariant, $\kpot$ can be chosen to depend only on $v$. Hence, the matrix that represents $\sform$ in these coordinates takes the form
	\begin{align*}
		\tilde{\sform} = \begin{bmatrix}
			0 & \vdots & \Hess_v(\kpot) \\
			\cdots & \cdots & \cdots \\
			-\Hess_v(\kpot) & \vdots & 0
		\end{bmatrix},
	\end{align*}
	with 
	\begin{align*}
		\Hess_v(\kpot) := [\kpot_{jk}]_{j,k=1}^{n,n}, \qquad \kpot_{jk}=\frac{\partial^2 \kpot}{\partial v_j \partial v_k}, 1 \leq j,k \leq n.
	\end{align*}
	the Hessian of $\kpot$ relative to $u$. The metric $\metric = \sform(\cdot,\cstruct\cdot)$ is then given locally by the matrix
	\begin{align} \label{eq_metric_kh}
		\tilde{\metric} = \tilde{\sform} \tilde{\cstruct} = \begin{bmatrix}
			\Hess_v(\kpot) & \vdots & 0\\
			\cdots & \cdots & \cdots \\
			0 & \vdots & \Hess_v(\kpot)
		\end{bmatrix}.
	\end{align}
	where $\tilde \sform, \tilde \metric$ are local matrix representations of $\sform, \metric$ in toric holomorphic coordinates, respectively. Note that, since $\metric$ defines a metric, \eqref{eq_metric_kh} must be positive definite, which means that $\kpot$ is strictly convex.

	Now, turning to action-angle coordinates, we can describe $\mfld^\circ$ as
	\begin{align} \label{eq_aa_coords}
		\mfld^\circ \cong \poly^\circ \times \torus^n = \{(u,\theta): u\in\poly \subset \reals^n, \theta \in \reals^n / \ints^n \}.
	\end{align}
	In these coordinates, $\tact$ takes the form
	\begin{align*}
		\tact(t)(u,\theta) = (u,\theta+t), \qquad t \in \torus^n,
	\end{align*}
	and the Kähler form $\sform = \sum_{j} du_j \wedge d\theta_j$ is represented by the matrix
	\begin{align} \label{eq_sform_aa}
		\tilde{\sform} = \begin{bmatrix}
			0 & \vdots & I\\
			\cdots & \cdots & \cdots \\
			-I & \vdots & 0
		\end{bmatrix}.
	\end{align}
	The crucial point is that we can find a so-called \emph{symplectic potential}
	\begin{align} \label{eq_spot}
		\sgen \in C^\infty(P^\circ)
	\end{align}
	which is the Legendre transform of $\kpot$ in the following sense:
	\begin{align} \label{eq_legendre_transform}
		\kpot(v) + \sgen(u) = \sum_{j}\ppd{\kpot}{v_j}(v) \cdot \ppd{\sgen}{u_j}(u),\text{ evaluated at } u = \ppd{\kpot}{v} \text{ or } v = \ppd{\sgen}{u}.
	\end{align}
	Thus, one has that 
	\begin{align} \label{eq_coordinate_change}
		u = \frac{\partial \kpot}{\partial v}(v), \qquad v = \frac{\partial \sgen}{\partial u}(u).
	\end{align}
	In action-angle coordinates, $J$ can be expressed in terms of $\sgen$ as
	\begin{align} \label{ga_cstruct}
		\hat{\cstruct} = \begin{bmatrix}
			0 & \vdots & -\Hess_u(\sgen)^{-1}\\
			\cdots & \cdots & \cdots \\
			\Hess_u(\sgen) & \vdots & 0
		\end{bmatrix},
	\end{align}
	where $\hat \cstruct$ is the local matrix representation of $\cstruct$ in action-angle coordinates, and
	\begin{align*}
		\Hess_u(\sgen) = [\sgen_{j,k = 1}^{n,n}], \qquad \sgen_{jk}=\frac{\partial^2 \sgen}{\partial u_j \partial u_k}, \qquad 1 \leq j,k \leq n,
	\end{align*}
	is the Hessian of $\sgen$ relative to $x$. The metric $\metric = \sform(\cdot,\cstruct \cdot)$ is such that
	\begin{align}\label{eq_metric_aa}
		\hat{\metric} = \hat{\sform} \hat{J} = \begin{bmatrix}
			\Hess_u(\sgen) & \vdots & 0\\
			\cdots & \cdots & \cdots \\
			0 & \vdots & \Hess_u(\sgen)^{-1}
		\end{bmatrix},
	\end{align}
	where $\hat \sform, \hat \metric$ are local matrix representations of $\sform, \metric$ in action-angle coordinates, respectively.

\section{Coherent State Transforms} \label{sec_lifting}
Let $\pol_0$ be a polarization on the quantizable symplectic manifold $(\mfld, \sform)$, as introduced in \secref{sec_gq}. Given $H \in C^\ra(\mfld)$, we now discuss how to lift the action of the complex flow of the Hamiltonian vector field $X_H$, as defined in \thmref{thm_complex_flow}, to the quantum bundle. That is, assuming that there are no convergence issues for the relevant values of $\tau \in \cmplx$, we wish to find a map
\[V_\tau: \hilb^{\text{hf}}_{\pol_0} \to \hilb^{\text{hf}}_{\pol_\tau},\]
where $\pol_\tau = e^{\tau \lied{X_H}}\pol_0$ is as in \eqref{eq_evolved_pol} and $\hilb^{\text{hf}}_{\pol_0}, \hilb^{\text{hf}}_{\pol_\tau}$ are as in %\eqref{eq_reduced_hilb}
\eqref{eq_reduced_hilb_whf}. % depending on whether we are considering the half-form correction or not.

This map will be a \emph{GCST (generalized coherent state transform)}, or \emph{KSH (Konstant-Souriau-Heisenberg) map} associated to $H$ (see \cite{kirwin_complex_2013}),
\begin{equation} \label{eq_cst_def}
U_s := \left ( e^{-\frac{i}{\hbar} \tau \prqhf{H}} \circ e^{\frac{i}{\hbar} \tau \q{H}} \right )\big|_{\tau = is},
\end{equation}
where $\prq{H}$ corresponds to the prequantization of $H$ (as in \eqref{eq_prq_whf}) and $\q{H}$ is the quantization of $H$ (see \remref{rem_full_quantization}). The exponential of an operator is as in \defref{def_general_lie_series}, assuming there are no convergence issues. 

From Section 4.4 and Remark 4.26 of \cite{kirwin_complex_2013}, we see that:
\begin{thm} \label{thm_gcst_asymp_unitarity}
	Let $\pol_0$ be a toric Kähler polarization on the quantizable toric symplectic manifold $(\mfld, \sform)$ with associated moment map $\mu$. Let $H \in C^\ra(\mfld)$ be toric and strictly convex as a function of $\mu$, and let $U_s$ be the GCST associated to $H$ as in \eqref{eq_cst_def}. Then $U_{s}$ is asymptotically unitary as $\hbar \to 0$.
\end{thm}
\begin{rem} \label{rem_gcst_interpretation}
	We can interpret $U_s$ as follows: the factor $e^{\frac{i}{\hbar} \tau \q{H}}$ corresponds to quantum evolution while the factor $e^{-\frac{i}{\hbar} \tau \prqhf{H}}$ is associated to an exact lifting of classical evolution in complex time. Thus, one can interpret the lack of unitarity of this map as due to a difference between the two types of evolution. This will play a role in the interpretation of the results in \secref{sec_conclusion}.
\end{rem}

% \begin{rem}
% 	Let $A \subseteq C^\infty(\mfld) \otimes \cmplx$. Then, for $f \in A$, $e^{-\frac{i}{\hbar} \tau \prq{H}}: \hilb{\pol_0} \to \hilb_{\pol_\tau}$ and $e^{\frac{i}{\hbar} \tau \q{H}}: \hilb{\pol_0} \to \hilb_{\pol_0}$. 
% 	In light of Schur's lemma, the representation on $\hilb_{\pol_\tau}^Q$ obtained as
% 	\begin{align*}
% 		g \mapsto V_\tau^f \circ \rep_{\pol_0}(g)\circ (V_{\tau}^f)^{-1}
% 	\end{align*}
% 	is not a $*$-representation or in other words, the discrepancy between classical and quantum evolution in the two factors of $V_f$ may spoil the star relations of the representation of $A$ in $\hilb_{\pol_\tau}^Q$
% \end{rem}

% Schur's lemma
% \begin{lem}
% 	Let $(H,\rho_1)$ be two irreducible $*$-representations of the $*$-algebra $A$ and $V$
% 	\begin{align*}
% 		V:\hilb_1 \to \hilb_2
% 	\end{align*}
% 	a linear operator intertwining them. Then $V$ is projectively unitary.
% \end{lem}


% \begin{ex}
% In \cite{hall_segal-bargmann_1994}, Hall constructed a unitary transform for Lie groups of compact type $G$.

% \begin{align*}
% 	U:L^2(G, dx) &\to \holo L^2(G_\cmplx, d\nu(g)) \\
% 	U &= \left (\mathcal{C} \circ e^{\frac{\Delta}{2}} \right )
% \end{align*}

% For the specific case $G = \reals$, $M=T^*\reals \cong \reals^2$

% \begin{align*}
% U:L^2(\reals, dq) &\to \mathcal{H}L^2(\cmplx, e^{-p^2} dpdq) \\
% U &= \left (\mathcal{C} \circ e^{\frac{\Delta}{2}} \right ) \\
% \psi(q) &\mapsto \left (e^\frac{\Delta}{2} \right )(q+ip)
% \end{align*}
% For $H=\frac{p^2}{2}$, $X_H=p \pd{q}$ and thus $e^{\tau X_H} \big|_{t=i} = (q + t p) \big|_{t=i} = z$. Hence, 
% \begin{align*}
% 	\mathcal{C} = e^{iX_H}
% \end{align*}.

% Since $\prq{H} = i X_H - \frac{p^2}{2}$ , \[e^{-it\prq{H}}\big|_{t=-is} = \mathcal{C} \circ e^{-\frac{p^2}{2}}.\]
 
% Since $\q{H} = H(\q{p})$ and $\q{p} = -i\pd{q}$ ,

% \[e^{-it\prq{H}}\big|_{t=-is} = \mathcal{C} \circ e^{-\frac{p^2}{2}}.\]
% \[e^{it \q{H}}\big|_{t=-is} = e^\frac{\Delta}{2}.\]
% Thus, the Hall CST is equivalent the transform lifting the complex canonical transfrmation $e^{\tau X_{H}|_{\tau=i}}=e^{i p \pd{q}}$ 
% \begin{align*}
% 	\hilb_\sch^Q = \hilb_q^Q &\to^{C^{iH}} \hilb_z^Q=H_\fock^Q \\
% U &= \left (\mathcal{C} \circ e^{\frac{\Delta}{2}} \right ) = \left (e^{-it\prq{H}} \circ e^{it \q{H}}\right )\big|_{t=-is}
% \end{align*} 
% where $e^{-\frac{p^2}{2}}$ is absorbed into the averaged heat kernel measure.
% \end{ex}
\end{document}