\documentclass[notas.tex]{subfiles}

\begin{document}
\chapter{Introduction}

Since its initial discovery, the quantum Hall effect, both integer \cite{klitzing_new_1980} and fractional \cite{tsui_two-dimensional_1982}, has been the subject of intensive reseach. Though topology has taken a central role in the study of this phenomenon (it is a so-called topological phase of matter), its geometrical aspects have not been neglected. 

Early on, expressions for the Laughlin state \cite{laughlin_anomalous_1983} have been given for different geometries \cite{azuma_explicit_1994, haldane_periodic_1985,haldane_fractional_1983}. More recently, Haldane identified a hidden variational parameter corresponding to a geometric degree of freedom \cite{haldane_geometrical_2011} in the fractional quantum Hall effect. This has been further explored in \cite{qiu_model_2012}, where geometry-dependent families of wave functions corresponding to model quantum Hall states on homogeneous anisotropic surfaces are constructed. Johri et al. \cite{johri_probing_2016} treats the problem both numerically and heuristically for the filling fraction $\nu = \frac{1}{3}$, using perturbative methods with very small deformations. Other research on the geoemtry of the quantum Hall effect includes e.g. Gromov's description of a geometric analogue of quasiholes in \cite{gromov_geometric_2016}, among various other topics \cite{ferrari_fqhe_2014, gromov_investigating_2017, gromov_bimetric_2017, liu_geometric_2018, yang_three-body_2018}. In his work, Klevtsov and his collaborators have taken a more mathematical approach to the problem \cite{ klevtsov_quantum_2017, klevtsov_laughlin_2019}; his research on Laughlin states on Riemann surfaces \cite{klevtsov_laughlin_2019} is particularly noteworthy.

In the present work, we resort to techniques from geometric quantization in order to study the dependence of quantum Hall ground states on the geometry of the surface on which they are defined. This approach is particularly adequate to this problem since, as stated in the introduction to \secref{sec_geometry_dependence_overview}, quantum Hall ground states can be represented as holomorphic sections of a suitable line bundle. Also, for this reason, we can use generalized coherent state transforms in order to see how these states change as the surface undergoes a deformation. This deformation will, in turn, be given by the  the complex-time flow of a Hamiltonian vector field.

In \secref{sec_prelim}, we present the theory needed in order to study the dependence of quantum Hall states on the geometry of the surface. We recall aspects of symplectic, complex and Kähler geometry in \secref{sec_gm}. Then, we introduce geometric quantization in \secref{sec_gq}, which provides us with a systematic way of associating a quantum theory to a symplectic manifold satisfying certain conditions. In \secref{sec_complex}, we look at how to complexify the flow of a Hamiltonian vector field, and show how this flow can be used to deform a surface. In \secref{sec_lifting}, we introduce generalized coherent state transforms, which allow us to see the effect that the deformation of the surface has on the quantum Hall ground state wave functions. Finally, in \secref{sec_coordinates}, we introduce action-angle and toric holomorphic coordinate formalisms, under which explicit computations become feasible.

In \secref{sec_geometry_dependence}, we apply the formalism introduced in \secref{sec_prelim} in order to see how the quantum Hall ground states vary as a plane (\secref{sec_plane}) and a cylinder (\secref{sec_cyl}) undergo $S^1$-invariant deformations.
\end{document}