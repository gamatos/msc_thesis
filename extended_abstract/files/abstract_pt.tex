\documentclass[notas.tex]{subfiles} 				%tamanho das paginas: a4

\begin{document}
\section*{Resumo}
Quantização geométrica e transformadas generalizadas de estados coerentas são usadas de forma a determinar como as funções de onda de estados de vácuo no efeito de Hall quântico variam em função de uma deformação $S^1$-invariante da superfície onde estão definidas. Esta deformação é induzida pelo fluxo complexificado de um campo vetorial $S^1$-invariante. Examina-se tanto o caso de um plano infinito como o de um cilindro infinito, ambos vistos como variedades de Kähler tóricas. Conclui-se que, em ambos, a evolução não altera de forma significativa a estrutura dos estados no caso de uma única partícula e do efeito de Hall quântico inteiro. Para estados de Hall quânticos fraccionários, contudo, a estrutura polinomial é alterada de forma fundamental, possivelmente refletindo a forma como a interacção de Coulomb entre as partículas é afetada pela mudança da geometria. No caso do cilindro, no limite da deformação, as funções de onda aproximam estados distribucionais concentrados em certas folhas de Bohr-Sommerfeld. Comparam-se estes resultados com a literatura existente.

\bigskip
{\bfseries Keywords:} Quantização geométrica, fluxo Hamiltoniano complexificado, transformada de estados coerentes generalizada, efeito de Hall quântico, nível de Landau fundamental.
\end{document}